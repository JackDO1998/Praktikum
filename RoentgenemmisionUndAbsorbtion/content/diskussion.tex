\section{Diskussion}
\label{Diskussion}
In diesem Versuch sollte zunächst die Braggbedingung überprüft werden, dazu wurde der LiF-Kristall auf einen 
Winkel von $\Theta=14°$ eingestellt und in Schritten von 0,1° vermessen. Das Maximum der Reflektierten 
Strahlung wurde bei 28,2° gefunden, der theoretische Wert liegt bei dem doppelten des Eingestellten Winkels
von 14° also 28° es gibt also nur eine Abweichung von 0,71\%.
Im nächsten Schritt wurde dann zunächst die Full Width at half Maximum berechnet, dafür wurde zwischen den beiden Messwinkeln 
welche am nächsten an der berechneten mittleren Impulsdichte liegen linear interpoliert und der genaue Winkel errechnet.
Aus diesem wurde dann das Auflösevermögen der Appereatur berechnet, es liegt für die $K_{\alpha}$-Linie bei $A_{\alpha}=43.0752$ und für die 
$K_{\beta}$-Linie bei $A_{\beta}=48.5050$. Im nächsten Schritt wurden mithilfe von Energiewerten aus der
NIST Datenbank die Abschirmkonstante $\sigma_1=3.2924$ errehnet. Durch diese und die Energien $E_{K,\alpha}$ und $E_{K,\beta}$ 
war es möglich auch die Abschirmkonstanten $\sigma_2=24.342$, $\sigma_3=3.998$ zu berechnen. 

%Theoriewerte für Abschirmkonstanten einfügen

Im darauf folgenden Kapitel wurden dann Proben von Brom, Zirkonium, Zink, Strontium, Rubidium und Gallium 
vermessen und aus dem Winkel an dem die jeweilige K-Kante ihre Mitte erreicht, die Emmisionsenergien und 
Abschirmkonstanten bestimmt die Ergebnisse werden in der Nachfolgenden Tabelle mit den ebenfalls aus der NIST-Datenbank stammenden
Thoeriewerten verglichen:

\begin{table}
    \centering
    \label{tab:magnetfeld}
    \caption{Vergleich mit Theoriewerten}
    \sisetup{table-format=1.2}
    \begin{tabular}{S[table-format=3.2] S S S S S [table-format=3.2]}
      \toprule
      {Absorber} & {$E_K$[KeV]}& {$E_K$ Theoriewert[KeV]} & {Abweichung[\%]}\\
      \midrule
      {Brom      }& 13.43 & 13.50  &0.52\\
      {Zirkonium }& 17.81 & 18.00  &1.07\\
      {Zink      }&  9.60 &  9.65  &0.52\\
      {Strontium }& 16.06 & 16.10  &0.24\\
      {Rubidium  }& 15.12 & 15.20  &0.53\\
      {Gallium  } & 10.32 &  9.07  &14.0\\
      \bottomrule
    
    \end{tabular}
  \end{table}
  \newpage
  Zuletzt wurde dann noch das Mosley'sche Gesetz überprüft indem aus der Energie $E_K$ über die Rydbergfrequenz 
  die bekannte Größe der Rydbergenergie $R_{\infty}=0.0136\si[]{Kev}$ berechnet wurde die Ergebnisse mit den 
  Abweichungen zum aus der Versuchsantleitung zum Versuch 602 stammenden Theoriewert sind in der nachfolgenden
  \autoref{tab:moseley2} dargestellt:
  \begin{table}
    \centering
    \label{tab:moseley2}
    \caption{Überprüfung des Moseley-Gesetzes}
    \sisetup{table-format=1.2}
    \begin{tabular}{S[table-format=3.2] S S S S S [table-format=3.2]}
      \toprule
      {Absorber} &  {$R_{\infty}$ in $\si[]{KeV}$} & {Abweichuing in $\%$}\\
      \midrule
      {Brom      }   & 0.0137 &0.74\\
      {Zirkonium }   & 0.0140 &2.94\\
      {Zink      }   & 0.0138 &1.47\\
      {Strontium }   & 0.0139 &2.22\\
      {Rubidium  }   & 0.0139 &2.22\\
      {Gallium   }   & 0.0138 &1.47\\

      \bottomrule
    
    \end{tabular}
  \end{table}

Die geringen Abweichungen können z.B. durch unregelmäßigkeiten im Kristall und durch weitere Fehler an der Apperatur verursacht werden.
Die einzige auffällige Abwichung liegt mit 14\% bei der Absorbtionsenergie des Galliums hier könnte ein systematischer Messfehler oder
ein unentdeckter Rechenfehler der Grund sein. Ansonsten sind die Abweichungen der Energien meist weit unterhalb eines Prozentes und damit
sehr gut. Auch die Abweichungen bei der Überprüfung des Moseleyschen Gesetzes liegen alle im Bereich unter 3\%, das Moseley-Gesetz kann 
also als korrekt angesehen werden.
Am Ende kann gesagt werden das es sich um einen aufschlussreichen Versuch gehandelt hat.
\section{Literatur}
\label{Literatur}
1. TU Dortmund, Versuch 602 Röntgenemmision und Absorbtion\\
2. Demtröder, Wolfgang, 1995, Experimentalphysik 2, 6.Aufl., Berlin\\
3. NIST X-Ray Transition Database, Abgerufen 18.04.2021, von https://physics.nist.gov/PhysRefData/XrayTrans/Html/search.html


