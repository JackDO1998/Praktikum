\section{Diskussion}
\label{Diskussion}
In diesem Versuch sollte zunächst die Braggbedingung überprüft werden, dazu wurde der LiF-Kristall auf einen 
Winkel von $\Theta=14°$ eingestellt und in Schritten von 0,1° vermessen. Das Maximum der Reflektierten 
Strahlung wurde bei 28,2° gefunden, der theoretische Wert liegt bei dem doppelten des Eingestellten Winkels
von 14° also 28° es gibt also nur eine Abweichung von weniger als einem Prozent.
Im nächsten Schritt wurde dann zunächst die Full Width at half Maximum berechnet, dafür wurden deie Messwinkel 
benutzt an denen die Halbe Impulsdichte zum letzen mal nicht überschritten bzw. zum ersten mal wieder 
unterschritten wurden, somit ergibt sich für das daraus im nächsten Schritt berechnete Auflösevermögen der
Versuchsanordung eine untere Schranke, diese liegt bei  $A_\alpha=35.4835$. Alternativ hätte man den nicht
vermessenen Raum zwischen den Punkten die jeweils über bzw. unter der FWHM-Linie liegen linear nähern können
und so ein genaueres Ergebnis erhalten können. Im nächsten Schritt wurden mithilfe von Energiewerten aus der
NIST Datenbank die Abschirmkonstanten $\sigma_1=3.2924$, $\sigma_1=11.0843$, $\sigma_1=2.1265$ berechnet. Da 
diese rein auf Theoriewerten beruhen, sind sie selbst Theoriewerte und es ist kaum eine Abweichung zu berechnen.
Im darauf folgenden Kapitel wurden dann Proben von Brom, Zirkonium, Zink, Strontium, Rubidium und Gallium 
vermessen und aus dem Winkel an dem die jeweilige K-Kante ihre Mitte erreicht, die Emmisionsenergien und 
Abschirmkonstanten bestimmt die Ergebnisse werden in der Nachfolgenden Tabelle mit ihren Thoeriewerten verglichen:

\begin{table}
    \centering
    \label{tab:magnetfeld}
    \caption{Vergleich mit Theoriewerten}
    \sisetup{table-format=1.2}
    \begin{tabular}{S[table-format=3.2] S S S S S [table-format=3.2]}
      \toprule
      {Absorber} & {$E_K$[KeV]}& {$E_K$ Theoriewert[KeV]} & {Abweichung[\%]}\\
      \midrule
      {Brom      }& 12.50 & 13.50  &8.0\\
      {Zirkonium }& 16.50 & 18.00  &9.1\\
      {Zink      }&  9.29 &  9.65  &3.9\\
      {Strontium }& 15.24 & 16.10  &5.6\\
      {Rubidium  }& 14.51 & 15.20  &4.8\\
      {Gallium  } &  9.58 &  9.07  &5.6\\

      \bottomrule
    
    \end{tabular}
  \end{table}
Die geringen Abweichungen können z.B. durch unregelmäßigkeiten im Kristall, durch das mit $\Delta\Theta=0.1°$ 
nur recht grob vermessene Spektrum und durch weitere Fehler an der Apperatur verursacht werden.
Am Ende kann gesagt werden das es sich um einen aufschlussreichen Versuch gehandelt hat.
\section{Literatur}
\label{Literatur}
1. TU Dortmund, Versuch 602 Röntgenemmision und Absorbtion\\
2. Wolfgang Demtröder, Experimentalphysik 2\\
3. NIST X-Ray Transition Database 

