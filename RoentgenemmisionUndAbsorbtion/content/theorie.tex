\section{Zielsetzung}
\label{sec:zielsetzung}
Das im folgenden behandelte Experiment beschäftigt sich mit der Absorbtion und Emission von Röntgenstrahlung. Dabei wird zunächst die Bragg-Bedingung 
überprüft. Anschließend wird das Emissionsspektrum von Kupfer, sowie die Absorptionsspektren verschiedener Elemente wie Brom und Rubidium untersucht.
\section{Theorie}
\label{sec:theorie}
Die im Experiment verwendete Röntgenstrahlung wird durch eine Röntgenröhre erzeugt. In dieser werden in einer evakuirten Röhre Elektronen an einer Glühkathode erzeugt und anschließend zu einer Anode hin beschleunigt. Beim Auftreffen der Elektronen auf das Anodenmaterial wird von diesem Röntgenstrahlung emitiert.
Das Spektrum der emitierten Röntgenstrahlung ist stark vom verwendeten Anodenmaterial (in diesem Fall Kupfer) abhängig und wird in zwei verschiedene unterkategorien von Spektren unterteilt. Diese Spektren, zum einen das kontinuirliche, zum anderen das charakteristische, unterscheiden sich grundlegend in der Form der Entstehung der jeweiligen Strahlung, die vom zugehörigen Spektrum beschrieben wird. \\
Das kontinuirliche Emissionsspektrum beschreibt die sogenannte Bremsstrahlung. Diese entsteht durch die Abbremsung eines Elektrons im Coulombfeld eines Nukleons, wobei das Elektron einen Teil seiner (kinetischen) Energie im Zuge der Abbremsung abgibt. Bei diesem Vorgang wird ein Röntgenquant emittiert, dessen Energie der Energiedifferenz des Elektrons entspricht.
\begin{equation}
E_{ph}=hf=E_{0.kin}-E_{kin}
\end{equation}
Da das Elektron Situationsabhängig einen beliebigen Anteil seiner kinetischen Energie abgeben kann, ist dieser Teil des Spektrums kontinuirlich. Für den Grenzfall, bei dem das Elektron seine gesamte kinetische Energie abgibt, ergibt sich die maximale Energie, und somit die minimale Wellenlänge des Spektrums. Für diesen Fall lässt sich aus der Beziehung $\frac{hc}{\lambda_{min}}=Ue$ die minimale Wellenlänge
\begin{equation}
\lambda_{min}=\frac{Ue}{hc}
\end{equation}
mit der Beschleunigungsspannung U und dem Planckschen Wirkungsquantum h bestimmen. \\
Das sogenannte characteristische Spektrum entsteht, wenn das einfallende Elektron ein Atom derart ionisiert, dass eine Leerstelle in einer der inneren Schalen des Atoms entsteht. Diese Leerstelle wird von einem Elektron aus einer der äußeren Schalen eingenommen was zur Emission eines Röntgenquants führt, dessen Energie der Differenz der Bindungsenergien beider Schalen entspricht ($E=E_m-E_n$). \\
Da die einzelnen Energieniveaus und daraus folgend die zugehörige Röntgenstrahlung spezifisch für das verwendete Anodenmaterial ist, tritt diese Form der Röntgenstrahlung nur mit bestimmten diskreten Energiewerten auf, die sich im Spektrum durch scharfe Linien zeigt. Dieses Spektrum ist also nicht kontinuirlich und charakteristisch für das Anodenmaterial der verwendeten Röntgenröhre. \\
Die betreffenden Linien werden in der Form $K_\alpha$, $K_\beta$, $L_\alpha$ usw. bezeichnet. Wobei der Buchstabe K/L... die innere Schale bezeichnet, auf die das Elektron übergeht, während der griechische Buchstabe im Index auf die Herkunft des Elektrons, also die Schale aus der das Elektron übergesprungen ist schließen lässt.

