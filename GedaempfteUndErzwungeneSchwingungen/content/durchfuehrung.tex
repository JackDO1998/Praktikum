\section{Durchführung}
Für den Versuch wird ein RCL-Kreis mit regelbarem Widerstand, ein Generator sowie ein Oszilloskop zum Darstellen der Spannungen verwendet. Durch den Generator kann der Schwingkreis sowohl durch einen einzelnen Impuls angeregt als auch durch eine Sinusspanung gespeist werden. Die Frequenz der anregenden Spannung lässt sich am Generator regulieren.  \\
Für die erste Messreihe wird der Widerstand der Schaltung so eingeschaltet, dass eine Schwingung zustande kommt, wie im ersten Fall der Theorie beschrieben. Anschließend wird das System durch den Generator angeregt, sodass eine gedämpfte harmonische Schwingung wie in Abbildung \ref{fig:gedämpfte Schwingung} dargestellt eintritt. Nun wird die Amplitude in Abhängigkeit von der Zeit gemessen. Anschließend wird der Widerstand am RCL-Kreis reguliert, bis der aperiodische Grenzfall eintritt und der zugehörige Widerstandswert notiert. \\
Für die darauffolgende Messreihe wird am Generator eine Sinusspanung eingestellt die zusammen mit der Kondensatorspannung am Oszilloskop angezeigt wird. Dann kann unter Veränderung der Frequenz der anregenden Spannung sowohl die Amplitude als auch die Phasenverschiebung der Kondensatorspannung untersucht werden, um deren Frequenzabhängigkeit festzustellen. 
