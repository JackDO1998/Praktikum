\section{Diskussion}
In diesem Versuch sollte zunächst in \autoref{sec:abklingdauer} die Abklingdauer bestimmt 
werden. Das Ergebnis liegt hier bei $T_{ex} = 0.89\pm 0.08 \si[]{ms}$.
Es liegen keine Vergleichswerte vor das Ergebnisch scheint jedoch autentisch zu 
sein. Im nächsten Versuchsteil in \autoref{sec:apg} wurde dann der theoretische
widerstands Wert des aperiodeischen Grenzfalls berechnet und festgestellt das dieser
um 48,94\% vom experimentell ermittelten Wert abweicht. Im nächsten Kapitel
\autoref{sec:res} wurde dann die Resonanzüberhöhung bzw. die Güte der Schaltung ermittelt.
Der Experimentelle Wert liegt hier bei $121.5\pm 0.4$ und weicht damit um 186.5\% vom 
errechneten Wert, welcher bei $q=42.41\pm 0.18$ liegt, ab. Im darauf folgenden 
Abschnitt, \autoref{sec:phs} wurden dann die Frequenzen untersucht bei denen 
die Phasenverschiebung bei 45°,90° und 135° liegt die hier gemessenen werte weichen für
$\omega_1$ um 90.78\%, für $\omega_2$ um 77,9\% und für $\omega_{res}$ 74,59\% ab. 
Die großen Abweichungen bei allen Messwerten werden vorallem daran liegen das 
die Werte vom Oszilloskop nur ungenau abgelesen werden konnten und das diverse andere
Faktoren wie Innenwiderstände vernachlässigt wurden.

\section{Literatur}
1. TU Dortmund Versuch 354 Gedämpfte und Erzwungene Schwingungen\\

\section{Anhang}
Auf den nächsten Seiten sind die Originalmesswerte zu finden.