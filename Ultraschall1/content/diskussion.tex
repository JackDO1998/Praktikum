\section{Diskussion}
\label{sec:diskussion}
In diesem Versuch wurde zunächst in \autoref{sec:schallgeschwindigkeit} mittels des Impuls-Echo-Verfahrens
die Schallgeschwindigkeit in Acrylglas bestimmt, das Ergebnis liegt bei $c_{Acryl}=(2685.8\pm3.4)\si[]{\frac{m}{s}}$
und weicht damit vom wahrscheinlichen Theoriewert von $c_{th}=2760\si[]{\frac{m}{s}}$ um etwa 2,7\% ab. Dieser Theoriewert ist der Wert für
das von der Röhm GmbH unter dem Handelsnamen "Plexiglas" vertriebene Acrylglas es ist jedoch möglich das es sich 
bei dem untersuchten Block um Material eines anderen Herstellers mit eventuell abweichenden Eigenschaften 
handelt. Allerdings ist eine Abweichung von rund drei Prozent zum Theoriewert auch ein annehmbares Ergebnis.
Im nächsten Teil \autoref{sec:fehlstellen} wurde dann die Position von Fehlstellen im Acrylblock ermittelt.
Die Ergebnisse sind zusammnen mit den zuvor mit einem Messschieber bestimmten Theoriewerten und den zugehörigen Abweichungen
in den folgenden Tabellen \autoref{tab:pos2} und \autoref{tab:pos3} dargestellt:

\begin{table}
    \centering
    \label{tab:pos2}
    \caption{Errechnete und gemessene Positionen der Fehlstellen}
    \sisetup{table-format=1.2}
    \begin{tabular}{S[table-format=3.2] S S S S  [table-format=3.2]}
      \toprule
      {Fehlstelle Nr} &{Position gem. [mm]} & {Position US 1 [mm]} & {Abweichung [\%]}\\
      \midrule
      1 &{$$17.00\pm 0.17$$} & {$$18.801\pm 0.024$$ }&{$$10.60\pm $$}\\
      2 &{$$19.20\pm 0.17$$} & {$$20.412\pm 0.026$$ }&{$$ 6.30\pm $$}\\ 
      3 &{$$60.10\pm 0.17$$} & {$$60.700\pm 0.080$$ }&{$$ 1.00\pm $$}\\ 
      4 &{$$52.70\pm 0.17$$} & {$$53.310\pm 0.070$$ }&{$$ 1.16\pm $$}\\ 
      5 &{$$45.40\pm 0.17$$} & {$$46.060\pm 0.060$$ }&{$$ 1.46\pm $$}\\ 
      6 &{$$37.90\pm 0.17$$} & {$$38.540\pm 0.050$$ }&{$$ 1.70\pm $$}\\ 
      7 &{$$29.90\pm 0.17$$} & {$$30.750\pm 0.040$$ }&{$$ 2.90\pm $$}\\ 
      8 &{$$21.85\pm 0.17$$} & {$$23.098\pm 0.029$$ }&{$$ 5.70\pm $$}\\ 
      9 &{$$13.80\pm 0.17$$} & {$$15.041\pm 0.019$$ }&{$$ 9.00\pm $$}\\ 
      10&{$$ 5.80\pm 0.17$$} & {$$7.3860\pm 0.009$$ }&{$$27.00\pm $$}\\ 
      11&{$$58.10\pm 0.17$$} & {$$54.660\pm 0.070$$ }&{$$ 5.93\pm $$}\\

      \bottomrule
    
    \end{tabular}
  \end{table}

  \begin{table}
    \centering
    \label{tab:pos3}
    \caption{Errechnete und gemessene Positionen der Fehlstellen}
    \sisetup{table-format=1.2}
    \begin{tabular}{S[table-format=3.2] S S S S  [table-format=3.2]}
      \toprule
      {Fehlstelle Nr} & {Position gem. 2 [mm]}&{Position US 2 [mm]} &{Abweichung [\%]}\\
      \midrule
      1 &{$$61.20\pm 0.1$$} & {$$60.830\pm 0.080$$} &  {$$ 0.60\pm 0.20 $$} \\
      2 &{$$59.00\pm 0.1$$} & {$$59.220\pm 0.070$$} &  {$$ 0.38\pm 0.21$$}\\
      3 &{$$13.40\pm 0.1$$} & {$$14.369\pm 0.018$$} &  {$$ 7.20\pm 0.80$$}\\
      4 &{$$21.80\pm 0.1$$} & {$$22.561\pm 0.028$$} &  {$$ 3.50\pm 0.50$$}\\
      5 &{$$30.20\pm 0.1$$} & {$$28.201\pm 0.035$$} &  {$$ 6.62\pm 0.33$$}\\
      6 &{$$38.60\pm 0.1$$} & {$$39.080\pm 0.050$$} &  {$$ 1.24\pm 0.29$$}\\
      7 &{$$46.60\pm 0.1$$} & {$$47.140\pm 0.060$$} &  {$$ 1.15\pm 0.25$$}\\
      8 &{$$54.65\pm 0.1$$} & {$$55.060\pm 0.070$$} &  {$$ 0.75\pm 0.22$$}\\
      9 &{$$62.70\pm 0.1$$} & {$$62.710\pm 0.080$$} &  {$$ 0.02\pm 0.20$$}\\
      10&{$$70.70\pm 0.1$$} & {verdeckt           } &  {verdeckt} \\
      11&{$$11.40\pm 0.1$$} & {$$16.383\pm 0.021$$} &  {$$43.70\pm 1.30 $$}\\
      \bottomrule
    
    \end{tabular}
  \end{table}
  Bei der zweiten Messung welche von der Unterseite des Blocks stattgefunden hat, konnte die Position der Fehlstelle Nr.10 nicht ermittelt werden da diese von der weit größeren
  Fehlstelle Nr. 11 verdeckt wurde. Auffällig sind die sehr großen prozentualen Abweichungen beid Messung 1, Fehlstelle Nr. 10 und Messung 2, Fehlstelle Nr. 11, mit 27\% und 43,7\%,
  diese könnten durch den großen Durchmesser von Fehlstelle Nr.11 oder durch die Tatsache das die beiden Fehlstellen direkt übereinander liegen entsanden sein.

Als letztes wurde in \autoref{sec:auge} versucht ein Augenmodell mit dem Impuls-Echo-Verfahren zu vermessen. Hier war es jedoch kaum möglich
zu belastbaren Ergebnissen zu gelangen, da es sich um ein recht altes leicht beschädigtes Modell handelte,
dieses wieß bereits Risse in der Hornhaut auf und war zu Dicht, im bezug auf Ultraschallwellen, um ein Messbares 
Echo des Untergrundes zuzulassen sodass über die Position des Netzhaut Peaks nur gemutmast werden kann.
Am Ende kann gesagt werden das zumindest der erste Versuchsteil aufschlussreiche und recht genaue Ergebnisse mit Abweichungen
von nur wenigen Prozent gebracht hat.
\section{Literatur}
\label{sec:literatur}
1. TU Dortmund, Versuch US1 Grundlagen der Ultraschalltechnik\\
2. Elcometer Instruments GmbH, Geschwindigkeitstabelle für vordefinierte Materialien Abgerufen 25.04.2021 von https://www.xn--ultraschallprfung-f3b.com/schallgeschwindigkeiten/
\section{Anhang}
Auf der nächsten Seite finden sich die Originalmesswerte.