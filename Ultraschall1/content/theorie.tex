\section{Zielsetzung}
Bei dem im Folgenden protokollierten Versuch wurde mithilfe von Ultraschall die Tiefe mehrerer Fehlstellen eines Probekörpers bestimmt. Anschließend wurde durch das gleiche Verfahren ein Augenmodell vermessen. 
\section{Theorie}
\label{sec:theorie}
Schallwellen werden anhand ihres Frequenzbereiches kategorisiert. Menschen hören in einem Bereich zwischen 
$16\si[]{Hz}$ und $20 \si[]{kHz}$. Frequenzen oberhalb dieses Grenzwertes bis zu ca. $1 \si[]{GHz}$ werden als Ultraschall 
bezeichnet, während bei dem Frequenzbereich unterhalb der menschlichen Hörschwelle von Infraschall gesprochen 
wird. Schallwellen mit einer Frequenz größer als $1 \si[]{GHz}$ nennt man Hyperschall.  
Schallwellen bewegen sich mithilfe von Druckschwankungen fort, welche sich, in Flüssigkeiten und 
Gasen, durch eine longitudinale Welle der Form
\begin{equation}
p(x,t)=p_0+v_0Zcos(kx-wt)
\end{equation}
beschreiben lässt. Hierbei ist $Z=c\rho$ die akustische Impedanz des Mediums, in dem sich die Welle fortbewegt. Sie setzt sich aus der Dichte des Materials sowie der Schallgeschwindigkeit im Material zusammen. Letztere ist bei Schallwellen materialabhängig, da diese auf Schwankungen der Materialdichte basieren. Von diesem Unterschied abgesehen verhalten sich Schallwellen im Bezug auf Reflexions- und Brechungsphänomene ähnlich wie elektromagnetische Wellen. 
In Gasen und Flüssigkeiten hängt die Schallgeschwindigkeit von der Dichte und der Kompressibilität $\kappa$ ab
\begin{equation}
c=\sqrt{\frac{1}{\kappa \rho}}
\end{equation}
In Festkörpern bildet Schall zusätzlich zu der longitudinalen eine transversale Welle aus, deren Geschwindigkeit sich von der longitudinalen unterscheidet. Allgemein besitzt die Geschwindigkeit im Festkörper die Form
\begin{equation}
c_F=\sqrt{\frac{E}{\rho}}
\end{equation}
mit dem Elastizitätsmodul $E$ \\
Bei dem Übergang einer Schallwelle zwischen zwei Medien unterschiedlicher Impedanz wird ein Teil der Welle reflektiert. Der Reflexionskoeffizient $R$ beschreibt das Verhältnis von einfallender und reflektierter Intensität und lässt sich aus den Impedanzen gemäß
\begin{equation}
R=(\frac{Z_1-Z_2}{Z_1+Z_2})^2
\end{equation}
berechnen, woraus sich auch der transmitierte Anteil $T=1-R$ berechnen lässt.
Weiterhin muss berücksichtigt werden, dass Schallwellen während der Propagation Energie durch Absorption verlieren. Daher nimmt die Schallintensität exponentiell mit der zurückgelegten Strecke ab
\begin{equation}
I(t)=I_0e^{-\alpha t}
\end{equation}
mit einem materialspezifischen Absorptionskoeffizienten $\alpha$. \\
Bei der Erzeugung von Ultraschall bedient man sich dem piezoelektrischen Effekt. Wenn ein piezoelektrischer Kristall wie beispielsweise Quarz entlang einer polaren Achse in einem elektrischen Wechselfeld platziert wird, kann er zu Schwingungen angeregt werden, welche zur Emission von Ultraschallwellen führen. Dieser Effekt ist besonders groß, wenn die Anregungsfrequenz mit der Eigenfrequenz des Kristalls übereinstimmt. Umgekehrt kann ein solcher Kristall auch als Empfänger verwendet werden, der eingehende Ultraschallwellen registriert, da er von diesen ebenfalls zu Schwingungen angeregt wird. \\
Es existieren im wesentlichen zwei Verfahren zur Untersuchung von Werkstoffen mittels Ultraschall. Zum einen das Durchschallungsverfahren, bei dem ein Ultraschallimpuls durch die Probe geschickt und am anderen Ende registriert wird. Falls der Impuls eine Fehlstelle passiert hat, lässt sich dies an der Verringerung der Intensität infolge der Reflektion eines Teils der Welle registrieren. \\
Bei dem Impuls-Echo-Verfahren wird die Ultraschallsonde gleichzeitig als Sender und Empfänger verwendet. Falls es zu einer Reflexion an einer Fehlstelle kommt wird das Echo, also der reflektierte Anteil der Welle, am Empfänger gemessen. Da die Zeit zwischen Aussenden des Impulses und Empfangen des Echos gemessen wird, kann diese Methode im Gegensatz zur Durchschallungs-Methode auch Aufschluss über die Tiefe der Fehlstelle geben. Diese lässt sich aus der Formel
\begin{equation}
    \label{eq:weg}
s=\frac{1}{2}ct
\end{equation}
berechnen.
