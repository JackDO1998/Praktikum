\section{Diskussion}
\label{sec:Diskussion}
Besonders auffällig sind in diesem Protokoll sind die großen 
Abweichungen der empirisch bestimmten Güteziffern von den Theoriewerten,
hier sind die empirischen Größen um knapp das doppelte kleiner als die
Theoretischen Werte.  Dabei liegen aus unserer Sicht
keine offensichtlichen Rechenfehler vor. Im Weiteren ist auffällig das die Mechanische Leistungen 
sehr geringe Werte aufweisen. 
Die deutliche Diskrepanz zwischen der idealen Güteziffer, die sich aus den 
Temperaturen der Reservoire ergibt, und der deutlich geringeren realen Güteziffer, 
die aus den Messwerten bestimmt wurde, lässt sich auf die vielen Näherungen zurückführen, 
die für die ideale Wärmepumpe vorgenommen wurden. Zunächst wird für die ideale Güteziffer 
zum einen angenommen, dass der Kompressor adiabatisch arbeitet und zum anderen, dass der 
vorliegende Prozess reversibel ist. Beide Annahmen treffen in einem realen System natürlich 
nicht zu. Bei einer idealen Wärmepumpe wird von einer perfekten Isolierung ausgegangen, wohingegen 
bei der realen Wärmepumpe der Austausch von Wärme mit der Umgebung für Verluste sorgt. Weiterhin 
werden Leistungsverluste des Kompressors, sowie durch Reibung auftretende Verluste vernachlässigt. 
Diese Effekte führen im realen System zu der deutlich geringeren Güteziffer.


\section{Literatur}
1. TU Dortmund Versuch 206 Die Wärmepumpe\\
2. TU Dortmund Versuch 203 Verdampfungswärme und Dampfdruckkurve\\
3. Meschede Gerthsen Physik 2015