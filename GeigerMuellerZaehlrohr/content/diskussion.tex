\section{Diskussion}
\label{sec:diskussion}
In diesem Versuch sollten Verschiedene Eigenschaften eines Geiger-Müller-Zählrohres beschrieben werden, dazu 
wurde zunächst in \autoref{sec:characteristik} die Kennlinie \autoref{fig:teilchenzahl} 
des Verwendeten Gerätes ermittelt und an das Plateau mittels linearer Regression ein Polynom ersten Grades 
angepasst. Dieses Polynom hat wie erwartet eine geringe Steigung von etwa $1,22\%$, was bedeutetd das es noch 
gelegentlich zu Nachentladungnen kommt, es sich aber im allgemeinen um ein qualitativ hochwertiges Zählrohr handelt.
Das Plateau ist jedoch schlecht zu identifizieren und kann daher die Aussage über die qualität des Zählrohrs nur  schwer
verifizieren. Hier wurde es mit einer Länge von etwa 260 Einheiten angenommen und liegt 
zwischen $\SI{360}{V}$ und $\SI{620}{V}$ in der gewählten Darstellung. Als nächstes wurde die Totzeit in 
\autoref{sec:totzeit} zunächst mithilfe eines Oszilloskopes bestimmt \autoref{sec:totzeitO}. 
Der abgelesene Wert von $T\approx 100\mu s$ ist wie sich in \autoref{sec:totzeitZ} zeigt sehr ähnlich groß wie
der mittels der Zwei-Quellen-Metode errechnete bzw. gemessene. Um die Totzeit genauer zu bestimmen wurde wie 
bereits erwähnt die Zwei-Quellen-Methode \autoref{sec:totzeitZ} verwendet. Der errchnete Wert $T=0,96\pm0,04µs$
liegt zumindest in der gleichen Größenordnung wie der Messwert der im vorherigen Kapitel mittels Oszilloskop 
bestimmten Wertes. Der Literaturwert liegt bei $0,1ms$ was sich exakt mit der Messung am Oszilloskopdeckt und
von dem errechneten Wert um etwa 10\% abweicht. Die Oszilloskop-Methode eignet sich sehr gut um eine schnelle 
Abschätzung vorzunehmen, allerdings können hier sehr leicht ablesefehler passieren die zu groben ungenauigkeiten
führen. Die Zwei-Quellen-Methode ist zwar bedeutend aufändiger da zwei Quellen benötigt werden und drei Messungen 
durchgeführt werden müssen, dafür ist sie allerdings im Rahmen der Näherung präzise und es können nur Rechenfehler 
passieren. Da eine Totzeitbestimmung theoretisch nur einmal durchgeführt werden muss, ist es sinnvoll die zwar 
aufwändigere aber präzisere Methode der Zwei-Qellen zu nutzen.  An Ende wurde noch die Anzahl der Ladungen die 
von einem einzelnen in das Geiger-Müller-Zählrohreinfallenden Teilchen im Rahmen eines Teilchenschauers 
ausgelöst werden und den Messdraht erreichen. DieseZahlen scheinen für Teilchen der entsprechenden Energie 
durchaus realistisch. Im ganzen kann von einem erkentnisreichen gut gelungenen Versuch gesprochen werden.

\section{Literatur}
\label{sec:literatur}
1. TU-Dortmund, V703 Das Geiger-Müller-Zählrohr\\
2. Dieter Meschede, Gerthsen Physik 25.Aufl.
3. https://www.chemie.de/lexikon/Geigerz\%C3\%A4hler.html

\section{Anhang}
\label{sec:anhang}
Auf den folgenden Seiten finden sich die Originalmesswerte.