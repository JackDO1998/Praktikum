\section{Diskussion}
\label{sec:diskussion}
In diesem Versuch sollten Verschiedene Eigenschaften eines Geiger-Müller-Zählrohres beschrieben werden, dazu 
wurde zunächst in \autoref{sec:characteristik} die Kennlinie \autoref{fig:teilchenzahl} 
des Verwendeten Gerätes ermittelt und an das Plateau mittels linearer Regression ein Polynom ersten Grades 
angepasst. Dieses Polynom hat hatte wie erwartet eine geringe Steigung von etwa $1,22\%$. Das Plateau ist 
jedoch schlecht zu identifizieren. Hier wurde es mit einer Länge von etwa 260 Einheiten angenommen und liegt 
zwischen $\SI{360}{V}$ und $\SI{620}{V}$ in der gewählten Darstellung. Als nächstes wurde die Totzeit in 
\autoref{sec:totzeit} zunächst mithilfe eines Oszilloskopes bestimmt \autoref{sec:totzeitO}. Da das Oszilloskop
mit 100µs/DIV noch zu grob eingestellt um eine genaue Messung zu liefern konnte hier nur grob geschätzt werden.
Der geschätzte Wert von $T\approx 2µs$ ist wie sich in \autoref{sec:totzeitZ} zeigt etwa doppetlt so groß wie
der mittels der Zwei-Quellen-Metode errechnete bzw. gemessene. Um die Totzeit genauer zu bestimmen wurde wie 
bereits erwähnt die Zwei-Quellen-Methode \autoref{sec:totzeitZ} verwendet. Der errchnete Wert $T=0,96\pm0,04µs$
liegt zumindest in der gleichen Größenordnung wie der Messwert der im vorherigen Kapitel mittels Oszilloskop 
bestimmten Wertes. An Ende wurde noch die Anzahl der Ladungen die von einem einzelnen in das Geiger-Müller-Zählrohr
einfallenden Teilchen im Rahmen eines Teilchenschauers ausgelöst werden und den Messdraht erreichen. Diese
Zahlen scheinen für Teilchen der entsprechenden Energie durchaus realistisch. Im ganzen kann von einem 
erkentnisreichen gut gelungenen Versuch gesprochen werden.

\section{Literatur}
\label{sec:literatur}
1. TU-Dortmund, V703 Das Geiger-Müller-Zählrohr
2. Dieter Meschede, Gerthsen Physik 25.Aufl.

\section{Anhang}
\label{sec:anhang}
Auf den folgenden Seiten finden sich die Originalmesswerte.