\section{Versuchsdurchführung}
\subsection{Versuchsaufbau}
Für den Versuch wurde die Anordung gemäß Abbildung 3 verwendet. Die Ladung des Zählrohrdrahtes löst am Widerstand einen Spannungsimpuls aus, der am Kondensator entkoppelt wird. Anschließend wird der Impuls verstärkt und am Zähler registriert oder an einem Oszillographen sichtbar gemacht. Die $\beta$-Strahlen Quelle wurde derart auf das Zählrohr gerichtet, dass die Zählrate $100imp/s$ nicht übersteigt, um Abweichungen aufgrund der vergleichsweise hohen Totzeit eines Geiger-Müller-Zählrohres zu vermeiden.
\subsection{Messung der Charakteristik}
Zur Messung der Charakteristik wurde die Spannung in Intervallen von $10V$ erhöht, und die Zahl der Impulse pro 60s gemessen. Diese Zeitspanne wurde gewählt, um zu gewährleisten, dass die Zahl der Impulse in der Größenordnung $N=10000$  liegt, damit der Messfehler $\Delta N=\sqrt{N}$ ca. 1\% oder geringer ist. Außerdem wurde die Zählrohrspannung in Abständen von $\Delta U=50V$ gemessen.
\subsection{Messung der Totzeit}
\subsubsection{Oszillograph}
Die Totzeit kann, wenn auch nur ungenau, bestimmt werden, indem von einem Oszillographen die Zeitspanne zwischen dem ursprünglichen Impuls und dem ersten nachfolgenden Impuls bei bekannter Ablenkgeschwindigkeit des Kathodenstrahls agelesen wird.
\subsubsection{Messung mithilfe der zwei-Quellen-Methode}
Um eine Totzeitkorrektur zur erhalten wurde die Impulsrate erhöht indem der Abstand zum Zählrohr verringert wurde. Anschließend wurde zunächst über $120s$  die Zählrate der ersten Quelle gemessen, anschließend wurde eine zweite Quelle hinzugefügt und zuletzt die erste Quelle entfernt und jeweils über die gleiche Zeitspanne gemessen.