\section{Messwerte}
\label{sec:messwerte}
In diesem Kapitel sind alle Messwerte und deren Umrechnungen aufgeführt. Die Originalmesswerte 
sind im Anhang \autoref{sec:anhang} zu finden.
\subsection{Teilchenzahlen}
\label{sec:werteTeilchenzahlen}
In der nachfolgenden Tabelle \autoref{tab:werteTeilchenzahlen} sind die Teilchenzahlen die pro 
Zeitintervall von $t=\SI{60}{s}$ im Zählrohr registirert wurden aufgeführt.
\begin{table}
    \label{tab:werteStrom}
    \centering
    \caption{Gemessene Impulse pro Zeitintervall in Abhängingkeit von der Spannung}
    \sisetup{table-format=1.2}
    \begin{tabular}{S[table-format=3.2] S S   [table-format=3.2]}
      \toprule
      {$U$[V]} & {$N$[Imp]}\\
      \midrule
320  &   9672\\
330  &   9689\\
340  &   9580\\
350  &   9837\\
360  &   9886\\
370  &   10041\\
380  &   9996\\
390  &   9943\\
400  &   9995\\
410  &   9980\\
420  &   9986\\
430  &   9960\\
440  &   10219\\
450  &   10264\\
460  &   10174\\
470  &   10035\\
480  &   10350\\
490  &   10290\\
500  &   10151\\
510  &   10110\\
520  &   10255\\
530  &   10151\\
540  &   10351\\
550  &   10184\\
560  &   10137\\
570  &   10186\\
580  &   10171\\
590  &   10171\\
600  &   10253\\
610  &   10368\\
620  &   10365\\
630  &   10224\\
640  &   10338\\
650  &   10493\\
660  &   10467\\
670  &   10640\\
680  &   10939\\
690  &   11159\\
700  &   11547\\
\bottomrule
    
    \end{tabular}
  \end{table}
  \newpage

\subsection{Messwerte Totzeit}
\label{sec:werteTotzeit}
Nachfolgend sind die Messwerte zur errechnung der Totzeit nach der Zwei-Quellen-Mathode dargestellt.
\begin{center}
    $N_1=96041 \frac{Imp}{120 s}$\\
    $N_2=76518 \frac{Imp}{120 s}$\\
    $N_{1+2}=158479 \frac{Imp}{120 s}$
\end{center}
\subsection{Messwerte Strom}
\label{sec:werteStrom}
Alle fünf Minuten oder alle 50 Volt wurde zusätzlich zu Spannung und Teilchenzahl noch der 
Wert für den Strom notiert. Diese Werte sind in der nachfolgenden Tabelle \autoref{tab:werteStrom} dargestellt. Das
verwendete Messgerät besitzt eine ungenauigkeit von $\Delta I=\pm0,05µA$
\begin{table}
    \label{tab:werteStrom}
    \centering
    \caption{Messwerte des Stromes}
    \sisetup{table-format=1.2}
    \begin{tabular}{S[table-format=3.2] S S S S  [table-format=3.2]}
      \toprule
      {$U$[V]} & {$I$[µA]}\\
      \midrule
      350 &   {$$0.3 \pm 0.05$$}\\
      400	&   {$$0.4 \pm 0.05$$}\\
      450	&   {$$0.7 \pm 0.05$$}\\
      500	&   {$$0.8 \pm 0.05$$}\\
      550	&   {$$1.0 \pm 0.05$$}\\
      600	&   {$$1.3 \pm 0.05$$}\\
      650	&   {$$1.4 \pm 0.05$$}\\
      700	&   {$$1.8 \pm 0.05$$}\\
\bottomrule
    
    \end{tabular}
  \end{table}
  \newpage