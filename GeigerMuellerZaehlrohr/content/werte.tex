\section{Messwerte}
\label{sec:messwerte}
In diesem Kapitel sind alle Messwerte und deren Umrechnungen aufgeführt. Die Originalmesswerte 
sind im Anhang \autoref{sec:anhang} zu finden.
\subsection{Teilchenzahlen}
\label{sec:werteTeilchenzahlen}
In der nachfolgenden Tabelle \autoref{tab:werteTeilchenzahlen} sind die Teilchenzahlen die pro 
Zeitintervall von $t=\SI{60}{s}$ im Zählrohr registirert wurden aufgeführt.
\begin{table}
    
    \centering
    \caption{Gemessene Impulse pro Zeitintervall in Abhängingkeit von der Spannung}
    \sisetup{table-format=1.2}
    \begin{tabular}{S[table-format=3.2] S S   [table-format=3.2]}
      \label{tab:werteTeilchenzahlen}
      \toprule
      {$U$[V]} & {$N$[Imp/min]}\\
      \midrule
      320  &  {$ 9672  \pm  98$}\\
      330  &  {$ 9689  \pm  98$}\\
      340  &  {$ 9580  \pm  98$}\\
      350  &  {$ 9837  \pm  99$}\\
      360  &  {$ 9886  \pm  99$}\\
      370  &  {$ 10041 \pm 100$}\\
      380  &  {$ 9996  \pm 100$}\\
      390  &  {$ 9943  \pm 100$}\\
      400  &  {$ 9995  \pm 100$}\\
      410  &  {$ 9980  \pm 100$}\\
      420  &  {$ 9986  \pm 100$}\\
      430  &  {$ 9960  \pm 100$}\\
      440  &  {$ 10219 \pm 101$}\\
      450  &  {$ 10264 \pm 101$}\\
      460  &  {$ 10174 \pm 101$}\\
      470  &  {$ 10035 \pm 100$}\\
      480  &  {$ 10350 \pm 102$}\\
      490  &  {$ 10290 \pm 101$}\\
      500  &  {$ 10151 \pm 101$}\\
      510  &  {$ 10110 \pm 101$}\\
      520  &  {$ 10255 \pm 101$}\\
      530  &  {$ 10151 \pm 101$}\\
      540  &  {$ 10351 \pm 102$}\\
      550  &  {$ 10184 \pm 101$}\\
      560  &  {$ 10137 \pm 101$}\\
      570  &  {$ 10186 \pm 101$}\\
      580  &  {$ 10171 \pm 101$}\\
      590  &  {$ 10171 \pm 101$}\\
      600  &  {$ 10253 \pm 101$}\\
      610  &  {$ 10368 \pm 102$}\\
      620  &  {$ 10365 \pm 102$}\\
      630  &  {$ 10224 \pm 101$}\\
      640  &  {$ 10338 \pm 102$}\\
      650  &  {$ 10493 \pm 102$}\\
      660  &  {$ 10467 \pm 102$}\\
      670  &  {$ 10640 \pm 103$}\\
      680  &  {$ 10939 \pm 105$}\\
      690  &  {$ 11159 \pm 106$}\\
      700  &  {$ 11547 \pm 107$}
\bottomrule
    
    \end{tabular}
  \end{table}
  \newpage

\subsection{Messwerte Totzeit}
\label{sec:werteTotzeit}
Nachfolgend sind die Messwerte zur Errechnung der Totzeit nach der Zwei-Quellen-Mathode dargestellt.
\begin{center}
    $N_1=96041 \frac{Imp}{120 s}$\\
    $N_2=76518 \frac{Imp}{120 s}$\\
    $N_{1+2}=158479 \frac{Imp}{120 s}$
\end{center}
\subsection{Messwerte Strom}
\label{sec:werteStrom}
Alle fünf Minuten oder alle 50 Volt wurde zusätzlich zu Spannung und Teilchenzahl noch der 
Wert für den Strom notiert. Diese Werte sind in der nachfolgenden Tabelle \autoref{tab:werteStrom} dargestellt. Das
verwendete Messgerät besitzt eine Ungenauigkeit von $\Delta I=\pm0,05µA$
\begin{table}
    
    \centering
    \caption{Messwerte des Stromes}
    \sisetup{table-format=1.2}
    \begin{tabular}{S[table-format=3.2] S S S S  [table-format=3.2]}
      \label{tab:werteStrom}
      \toprule
      {$U$[V]} & {$I$[µA]}\\
      \midrule
      350 &   {$$0.3 \pm 0.05$$}\\
      400	&   {$$0.4 \pm 0.05$$}\\
      450	&   {$$0.7 \pm 0.05$$}\\
      500	&   {$$0.8 \pm 0.05$$}\\
      550	&   {$$1.0 \pm 0.05$$}\\
      600	&   {$$1.3 \pm 0.05$$}\\
      650	&   {$$1.4 \pm 0.05$$}\\
      700	&   {$$1.8 \pm 0.05$$}\\
\bottomrule
    
    \end{tabular}
  \end{table}
  \newpage