\section{Auswertung}
\label{sec:Auswertung}
In diesem Kapitel werden alle Mittelwerte und deren Fehler berechnet. 
Dazu wurde Python Numpy benutzt. Diese Mittelwerte sind die anzunehmenden, fehlerbehafteten Größen.

Der Mittelwert:
\begin{center}
  \begin{equation}
    \label{eq:Mittelwert}
  \bar{x}=\frac{1}{n}\sum\nolimits_{i=0} x_i
  \end{equation} 
\end{center}

Die Standardabweichung:
\begin{center}
  \begin{equation}
    \label{eq:standardabweichung}
  
    $\sigma=\sqrt{\frac{\sum(x_i-\bar{x})^2}{n-1}}$
  \end{equation}
\end{center}

Der Fehler des Mittelwertes:
\begin{center}
  \begin{equation}
    \label{eq:mittelwertfehler}
    \sigma_{\bar{x}}=\frac{\sigma}{\sqrt{n}}
  \end{equation}

  
\end{center}

Die Gaußsche Fehlerfortpflanzung:
\begin{center}
\begin{equation}
  \label{eq:gaussfehler}  
\sigma_x=\sqrt{(\frac{\partial f}{\partial x_1})^2\sigma_{x_1}^2+(\frac{\partial f}{\partial x_2})^2\sigma_{x_2}^2+...+(\frac{\partial f}{\partial x_n})^2\sigma_{x_n}^2}
\end{equation}
\end{center}
\subsection{Abmessungen und Spezifikationen der Proben}
\label{sec:abmessungen}
\subsubsection{Abmessungen Draht}
\label{sec:abmessungenDraht}
Bei der ersten vermessenen Probe handelt es sich um einen Kupferdraht mit folgenden Werten:
\begin{center}
    Länge: $L_D=\SI{1,37}{m}$\\
    Durchmesser: $d_D=1,05\pm0,01 \times 10^{-4} m$\\
    Wiederstand: $R_D=\SI{2,6\pm0,7}{\Omega}$
\end{center} 
Um den Wert für den Widerstandangeben zu können wurde die Probe in beide Richtungen 
vermessen und der Mittelwert nach \autoref{eq:Mittelwert} mit zugehörigem Fehler nach 
\autoref{eq:mittelwertfehler} berechnet.
\subsubsection{Abmessungen der Folie}
\label{sec:abmessungenFolie}
Die zweite vermessene Probe ist eine Kupferfolie mit diesen Werten:
\begin{center}
    Länge: $L_F=25,0\pm1,0\times 10^{-3} m$\\
    Breite: $b_F=24,0\pm1,0\times 10^{-3} m$\\
    Dicke: $d_F=2,7\pm0,1\times 10^{-5} m$
\end{center} 
\subsubsection{Die Hallspannung}
Für die Folie aus \autoref{sec:abmessungenFolie} lässt sich die Hallspannung ohne Störung mit 
\autoref{eq:hallspannung} berechnen. Es folgt also:


\subsection{Die mikroskopischen Leitfähigkeitsparameter}
\label{sec:leitfaehigkeitsparameter}
In diesem Kapitel sollen einige Größen berechnet werden um die Materialeigenschaften der Probe
auf mikroskopischer Ebene zu spezifizieren.
\subsubsection{Ladungsträger pro Volumen $n$}
\label{sec:ladungstraegerVolumen}
Die Anzahl der Ladungsträger pro Volumeneinheit lässt sich mit \autoref{eq:elektronenzahl} berechen, da 
für unterschiedliche Magnetfeldflüsse unterschiedliche Hallspannungen gemessen werden ergibt sich auch
eine Reihe von Anzahlen für Ladungsträger pro Volumen. Da es aber nur eine korrekte Zahl geben kann wird hier
der Mittelwert nach \autoref{eq:Mittelwert} und dessen Fehler nach \autoref{eq:mittelwertfehler} angeben:
\begin{center}
    $n=6,26\pm0.35 \times 10^{27} \frac{1}{m^3}$
\end{center}
\subsubsection{Ladungsträger pro Atom $z$}
\label{sec:ladungstraegerAtom}
\subsubsection{Die mittlere Flugzeit $\bar{\tau}$}
\label{sec:flugzeit}
Aus \autoref{eq:widerstand} folgt für eien Runden Draht mit $Q=\frac{\pi d_D^2}{4}$ sofort:
\begin{equation}
    \label{eq:tau}
    \bar{\tau}=\frac{8m_0L}{e_0^2n R \pi d_{D}^2}
\end{equation}
\begin{center}
    $\Rightarrow \bar{\tau}=(1.79\pm0.11) \times10^{-12} As^2$
\end{center}

\subsubsection{Die mittler Driftgeschwindigkeit $\bar{v_d}$}
\label{sec:driftgeschwindigkeit}
Mit \autoref{eq:drift} und $j=1000000\frac{A}{m^2}$:
\begin{center}
    $\bar{v_d}=-\frac{j}{ne_0}$\\
    $\Rightarrow \bar{v_d}=-1,0 \pm 0,06 \times 10^{-3} \frac{m}{s}$
\end{center}
\subsubsection{Die Beweglichkeit $\mu$}
\label{sec:beweglichkeit}
Nach der Rechenvorschrift \autoref{eq:mu} folgt mit dem Ergebnis aus \autoref{sec:flugzeit}:
\begin{center}
    $\mu=0.158\pm0.009 \frac{A^2s^3}{kg} $
\end{center}
\subsubsection{Die Totalgeschwindigkeit v}
\label{sec:totalgeschwindigkeit}
Die totale Driftgeschwindigkeit $\vert v \vert$ ergibt sich mit \autoref{eq:vtot} und dem Ergebnis aus 
\autoref{sec:ladungstraegerVolumen} zu:
\begin{center}
    $\vert v \vert=(6.60\pm0.12)\times10^5 \frac{m}{s}$
\end{center}

\subsubsection{Die mittler freie Wellenlänge $\bar{l}$}
\label{sec:wellenlaenge}
Die materialspezifische freie Wellenlänge ergibt sich über \autoref{eq:freieWellenlaenge} mit den Ergebnissen 
aus \autoref{sec:totalgeschwindigkeit} und \autoref{eq:tau} zu:
\begin{center}
    $\bar{l}=(1.18+/-0.05)\times 10^{-6} m$
\end{center}
\subsection{Löcher- oder Elektronenleitung}
\label{sec:leiterart}
\subsection{Flussdichte $B$ des verwendeten Magneten}

