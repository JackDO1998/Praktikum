\section{Diskussion}
\label{Diskussion}
Bei diesem Versuch kommt es auf hohe Präzision bei den Messungen an. Da die zu messenden Größen sehr 
klein sind, werden auch scheinbar unbedeutende Größen wie Übergangswiderstände zwischen Laborsteckern
und den zugehörigen Buchsen relevant. Trotzdem sind die Messwerte und die daraus errechneten Größen durchaus
realistisch und nah an Literaturwerten. Die Anzahl der zur elektrischen Leitung nutzbaren Elektronen 
\autoref{sec:ladungstraegerAtom} $z$ liegt bedeutend näher an eins als an einer anderen Zahl damit lässt 
sich diese Zahl nicht "präzise" nennen, sie gibt jedoch einen eindeutigen Hinweis darauf das Kupfer nur 
über ein zur elektrischen Leitung nutzbaren Elektrons verfügt. Die mittlere Flugzeit \autoref{sec:flugzeit}
liegt mit ihrer Größenordung von $10^{-12}$ im Bereich von Größenordnungen die auch Literaturwerte von 
anderen Metallen aufweisen. Die mittlere Driftgeschwindigkeit \autoref{sec:driftgeschwindigkeit} passt 
von der Größenordnung zu den Erwartungen, ist jedoch entgegen der Erwartung eine Negative Größe. Es ist 
anzunehmen das ein Vorzeichenfehler im Verlauf der Berechnungen passiert ist, welcher jedoch nicht sofort
offen sichtbar ist. Die Beweglichkeit µ \autoref{sec:beweglichkeit} hat einen annehmbaren und realistischen
Wert. Die totale Geschwindigkeit v \autoref{sec:totalgeschwindigkeit} passt von der Größe zur alltäglichen 
Erfahrung das sich elektrische Signale in Kupfer nahezu instantan ausbreiten. Die mittlere freie Wellenlänge
\autoref{sec:wellenlaenge} ist ebenfalls nahe an der anzunehmenden Größenordnung von etwa $10^{-5}$ bis 
$10^{-6}$ die für das von Kupfer ausgebildete fcc-Gitter zu erwarten ist. Ob es sich bei der Probe um ein
Löcher- oder Elektronenleiter handelt \autoref{sec:leiterart} kann nur abgeschätzt werden, da die Messung 
der Hallspannung nur an einer einzelnen Probe durchgeführt wurde und daher keine Vergleichswerte vorliegen.
Der verwendete Elektromagnet \autoref{sec:verwendeterMagnet} verhält sich unter Änderung des Spulenstromes, 
wie in \autoref{fig:magnet} gut sichtbar nahezu linear, was für realistische Messwerte die Flussdichte 
betreffend spricht. Im ganzen betrachtet kann also von einem gelungen Experiment gesprochen werden auch
wenn es einige Quellen für Messfehler gab und keine sich als präzise bezeichnen lässt.

\section{Literatur}
\label{Literatur}
1. TU Dortmund, Versuch 311 Halleffekt\\
2. Wolfgang Demtröder, Experimentalphysik 2\\
3. Particle Data Groub, Particle Physics Booklet

\section{Anhang}
\label{sec:Anhang}
Auf den nächsten beiden Seiten befinden sich die Scans der Orginalwerte. 