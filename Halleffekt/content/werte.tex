\section{Werte}
\label{sec:werte}
In diesem Kapitel werden alle verwendeten Werte und Konstanten dargestellt. Im  Anhang \autoref{sec:anhang}
findet sich zudem ein Scan der Orginaldaten.
\subsection{Messwerte}
Die nachstehende Tabelle \autoref{tab:probeconst} zeigt die gemessenen Hallspannungen $U+$ und $U-$.
\begin{table}
    \centering
    \label{tab:probeconst}
    \caption{Daten für eine Kupferprobe}
    \sisetup{table-format=1.2}
    \begin{tabular}{S[table-format=3.2] S S S  [table-format=3.2]}
      \toprule
      {$I$[A]} & {$U+$[µV]}& {$U-$[µV]}\\
      \midrule
        0.500  &    2.000  &   5.600\\
        1.000  &    2.000  &   6.600\\
        1.500  &    1.800  &   7.200\\
        2.000  &    1.100  &   8.000\\
        2.500  &    0.200  &   8.900\\
        3.000  &   -0.500  &   9.800\\
        3.500  &   -1.500  &   10.600\\
        4.000  &   -2.100  &   11.200\\
        4.500  &   -2.600  &   11.500\\
\bottomrule
    
    \end{tabular}
  \end{table}

Diese Tabelle \autoref{tab:magnetfeld} zeigt die gemessenen Feldstärken $B$ im Elektromagnet. Um $B+$ und $B-$
zu erhalten wurde der Magnet umgepolt.
  \begin{table}
    \centering
    \label{tab:magnetfeld}
    \caption{Daten des Magnetfeldes}
    \sisetup{table-format=1.2}
    \begin{tabular}{S[table-format=3.2] S S S  [table-format=3.2]}
      \toprule
      {$I$[A]} & {$B+$[T]}& {$B-$[T]}\\
      \midrule
0.500  &    2.000  &   5.600\\
1.000  &    2.000  &   6.600\\
1.500  &    1.800  &   7.200\\
2.000  &    1.100  &   8.000\\
2.500  &    0.200  &   8.900\\
3.000  &   -0.500  &   9.800\\
3.500  &   -1.500  &   10.600\\
4.000  &   -2.100  &   11.200\\
4.500  &   -2.600  &   11.500\\
\bottomrule
    
    \end{tabular}
  \end{table}
\subsection{Konstanten}
\label{sec:Konstanten}
\begin{center}
    $N_A=6.02214076 \times 10^23 \frac{1}{mol}$\\
    $e_0=1.602176634\times10^{-19} C$\\
    $m_0=9.1093837015\times 10^{-31} kg$\\
    $h=6.62607015\times 10^{-34}$
\end{center}