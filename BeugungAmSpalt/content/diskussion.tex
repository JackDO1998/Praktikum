\section{Diskussion}
\label{sec:diskussion}
In diesem Versuch ging es darum die Beugungsbilder des einfachen und des doppelten Spaltes zu bestimmen.
Dazu wurde zunächst der Dunkelstrom des Detektors bestimmt, dieser liegt mit einem Wert von $I_D=10\si[]{nA}$
in einem realistischen bereich wenn man bedenkt das die Messwerte im bereich von einigen $\mu A$ und somit um
3 Größenordnungen öher liegen. Anschließend wurde dann das Beugungsbild am Einzelspalt bestimmt. Die Messwerte
sind in \autoref{fig:es} aufgetragen und scheinen in erster Näherung dem bekannten Beugungsmuster eines Einzelspaltes
zu ähneln, einzig das Maximum ist sehr schmal. Sobald dann versucht wird das entsprechende Gesetz anzupassen 
fällt auf das dies kaum möglich ist. So musste aufwändig nach geeigneten Startwerten gesucht werden um überhaupt zu
einem Ergebnis zu gelangen. Am anpassen der Fouriertransformierten
scheint trotz sehr vielen Durchläufen der Ausgleichslgorithmus zu scheitern. 
Wie leicht zu sehen ist passt der Fit auch kaum zu den Messwerten. Die freien Parameter
können mit den Theorieangaben welche hier vom Hersteller des Spaltes stammen verglichen werden. So weicht
der experimentell gefundene Wert von $b=0.855\pm 0.085\si[]{mm}$ um etwa 570 \% vom Herstellerwert $b=0.15\si[]{mm}$
ab. Im nächsten Schritt \autoref{sec:doppelspalt} wurde dann das Beugungsmuster eines Doppelspaltes vermessen. 
In \autoref{fig:ds} ist sofort zu sehen das scheinbar an der entscheidenen Stelle um das Maximum herum nicht genau
genug gemessen wurde. Hier wurde der Detektor in $0.1\si[]{mm}$ Schritten bewegt es hätte jedoch scheinbar noch gensauer
sein müssen. Daher ist die entsprechende Augleichskurve auch eher ungenau. Diese Annahme manifestiert sich 
in den freien Parametern $s=(0.8023\pm 0.004)\si[]{mm}$ und $b=(0.0819\pm 0.004)\si[]{mm}$ deren theoretische pendants
liegen bei $s=0.9\si[]{mm}$ und $b=0.15\si[]{mm}$ und weichen somit um 8\% bzw. 83\% ab. Im ganzen kann 
also gesagt werden das der Versuch nicht oder nur recht ungenau gelungen ist. Fehlerquellen könnten darin liegen das
im Raum noch Schreibtischlampen eingeschaltet waren sowie die Tür offen stand. Zudem ist der Detektor nicht 
punktförmig sondern räumlich ausgedehnt was bei verschiebungen von nur $0.1\si[]{mm}$ von bedeutung sein könnte.
Wahrscheinlicher ist bei derart großen Abweichungen allerdings ein systematischer Fehler beim Messen, so können 
ablesefehler passieren wenn die Skala des Messgerätes geändert wurde, oder in der rechnerrischen Auswertung der
Messwerte.
Der Wert der relativen Abweichung ergibt sich über:
\begin{center}
    $F=\frac{Messwert-Theoriewert}{Theoriewert} 100\%$
\end{center}
\section{Literatur}
\label{sec:literatur}
[1] TU-Dortmund, V406 Beugung am Spalt

\section{Anhang}
\label{sec:Anhang}
Auf den nächsten Seiten ist ein Scan der Originalmesswerte zu finden.