\section{Auswertung}
\label{sec:auswertung}
In diesem Kapitel sollen die aufgenommenen Messwerte ausgewertet und in Beziehung gebracht werden.

\subsection{Der Einfachspalt}
\label{sec:einfachspalt}
Zunächst wurde eine Dunkelstrommesseung durchgeführt, dazu wurde im abgedunkelten Raum bei ausgeschaltetem
Laser der Detektorstrom $I_D$ gemessen. Er belauft sich auf:
\begin{center}
    $I_D=10\times 10^{-9}\si[]{A}$
\end{center}
Zudem wurde noch der Abstand zwischen beugendem Element und Detekor zu $z=1.05\si[]{m}$ bestimmt.
Der Beugungswinkel $\phi$ ergibt sich über die Beziehung:
\begin{center}
    $\phi=arctan(\frac{x_1}{z})$
\end{center}
wobei $x_1$ den Abstand vom Intensitätsmaximum beschreibt.
An die Messwerte wurde