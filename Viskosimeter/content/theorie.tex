\section{Zielsetzung}
\label{sec:zielsetzung}
Im vorliegenden Experiment wird ein Höppler-Viskosimeter verwendet, um die  tembaraturabhängige Viskosität von Wasser zu bestimmen.
\section{Theorie}
\label{sec:theorie}
Auf einen Festkörper, der sich durch eine zähe Flüssigkeit bewegt, wirken im wesentlichen drei Kräfte. Zunächst wird der Körper durch die Gewichtskraft $F_G=mg$ konstant in Richtung Boden beschleunigt. Dem entgegen wirkt die Auftriebskraft $F_A$, welche proportional zum verdrängten Volumen ist und darüber hinaus von der Dichte des verdrängten Mediums abhängt. Darüber hinaus steht der (bewegte) Körper unter Einfluss der Reibungskraft $F_R$, welche immer der Bewegungsrichtung entgegen wirkt und vom Betrag der Geschwindigkeit abhängt. Unter geeigneten Bedingungen lässt sich die Reibungskraft für eine Kugel mit Radius $r$, die sich in einer unendlich ausgedehnten Flüssigkeit mit der Geschwindigkeit $v$ fortbewegt, nach der Stokesschen-Beziehung
\begin{equation}
F_R=6\pi \eta rv
\end{equation}
berechnen. $\eta$ ist hier die Viskosität der Flüssigkeit, eine materialspezifischer Wert, der stark Temparaturabhängig ist. Diese Abhängigkeit läässt sich für die meisten Flüssigkeiten mithilfe eines exponentiellen Verlaufs der Form
\begin{equation}
    \label{eq:theoriekurve}
\eta (T) = Ae^{B/T}
\end{equation}
darstellen. Da die Reibungskraft proportional zur Geschwindigkeit wächst, während die beschleunigende Kraft konstant ist, wird nach einer bestimmten Zeit die Geschwindigkeit einen Grenzwert erreichen, für den die Reibungskraft betraglich identisch mit der Differenz aus Gewichts- und Auftriebskraft ist, sodass sich beschleunigende Kraft und Reibungskraft ausgleichen. Daher wird nach einer Beschleunigungsphase eine kontante Grenzgeschwindigkeit erreicht, mit der sich die Kugel fortbewegt. \\
Damit die Stokessche Beziehung git, muss sichergestellt werden, dass der Abstand der Kugel zur Rohrwand größer als die Ausdehnung der mitbewegten Flüssigkeitsschicht ist. Weiterhin kann einer geringer Abstand zur Rohrwand zu nicht reproduzierbaren Berührungen von Kugel und Wand führen, die Messungen verfälschen. Daher besitzt das Höppler-Viskosimeter einen leichten Neigungswinkel des Rohres, der in einer einfach zu reproduzierenden Bewegung der Kugel entlang der Rohrwand Resultiert. Unter diesen Vorraussetzungen lässt sich die Viskosität einer beliebigen Flüssigkeit gemäß folgender Beziehung bestimmen:
\begin{equation}
    \label{eq:eta}
\eta = tK(\rho_k - \rho_f)
\end{equation}
Hier bezeichnet $\rho_f$ die Dichte der Flüssigkeit und $\rho_k$ die Dichte der Kugel, sowie $t$ die gemessene Zeit, die die Kugel für die Strecke zwischen oberer und unterer Messchranke benötigt. Weiterhin wird zur Berechunng die Kugelkonstante $K$ benötigt, welche von verschiedenen Faktoren wie der Geometrie der Anordnung sowie den Größen aus der Stokesschen Beziehung abhängt. Sie ist nicht analytisch berechenbar und muss daher vor dem eigentlichen Versuch mithilfe einer Kalibrierflüssigkeit oder einer zweiten Kugel mit bekannter Konstante empirisch bestimmt werden.
