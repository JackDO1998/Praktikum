\section{Diskussion}
\label{sec:diskussion}
In diesem Versuch sollte die temperaturabhängige Größe der Viskosität von Wasser mit dem Viskosimeter 
nach Höppler untersucht werden. Dazu wurde in \autoref{sec:viskositaet1} zunächst die Viskosität von Wasser
bei 20,5 °C bestimmt. Der errechnete Wert von $\eta=(0.982\pm0.017)\si[]{mPa*s}$ weicht nur wenig von dem Literaturwert 
ab der etwa bei $\eta=1\si[]{mPa*s}$ liegt. Auf grundlage diese Wertes konnte dann in \autoref{sec:kugelkonstante1}
die Kugelkonstante $K_{gross}$ berechnet werden mithilfe welcher in \autoref{sec:viskositaet} die temperaturabhängige
Viskosität für Wasser bei einigen ausgesuchten Temperaturen errechnet werden konnte. Dazu wurden die Dichten und
Zeiten gemittelt um ein möglichst präzises Ergebnis zu erhalten. Anschließend wurden die Viskositäten für entsprechende
Temperaturen berechnet. Die Ergebnisse scheinen gut zu den Literaturwerten zu passen, leider konnte keine Tabelle
gefunden werden welche die Viskositäten für Wasser in 0,1 °C Schritten beschreibt daher seien hier nur einige Werte 
beispielhaft aufgeführt: Die Viskosität von Wasser liegt laut den Ergebnissen dieses Versuchs bei einer Temperatur
von etwa 35,7 °C bei $(0.006\pm 0.001) P$ der Literaturwert für 35 °C liegt bei $0.00719 P$ und damit nur knapp neben dem Fehlerband.
Der errechneteWert für etwa 59,2 °C liegt bei $0.005 P$ der Literaturwert für 60 °C bei $0.004666 P$ und damit ebenfalls
nur knapp außerhalb des Fehlers. Bei 88,9 °C ist der berechnete Wert bei $(0.004\pm0.001) P$ zu finden der Literaturwert
liegt hier sogar innerhalb des Fehlerbandes für 90 °C bei $0.003146 P$. Im allgemein kann also gesagt werden das 
die errechneten Werte durchaus realistisch sind. Mögliche Abweichungen sind mit dem großen Ablesefehler bei der
Temperatur und beim messen der Zeit zu erklären. Außerdem sind die jeweiligen Literaturwerte jeweils nur in der
Nähe der gemessenen Temperatur wodurch die Vergleichbarkeit schlechter, wenn auch nicht unzulässig wird. Im Ganzen
kann der Versuch als gelungen beschrieben werden.
\section{Literatur}
\label{sec:literatur}
1. TU-Dortmund, V107 Das Viskosimeter\\
2. https://www.internetchemie.info/chemie-lexikon/daten/w/wasser-dichtetabelle.php (07.02.2021) - Die Dichten von Wasser\\
3. https://www.chemie.de/lexikon/Wasser\_\%28Stoffdaten\%29.html (08.02.2021)- Die Viskositaeten
\section{Anhang}
\label{sec:anhang}
Auf den folgenden Seiten finden sich die Originalmesswerte.