\section{Diskussion}
\label{sec:diskussion}
In diesem Versuch werden die Elastizitätsmodule von verschiedenen Materialien gemessen bzw. errechnet.
Dazu wurden zunächst in \autoref{sec:dichte} die Dichten der Metalle bestimmt. Sie liegen wie in
\autoref{tab:eigenschaften} zu sehen für den ersten Stab bei $$\rho_1=(8.37\pm 0.09)\si[]{\frac{t}{m^3}}$$ für den zweiten Stabe 
bei $$\rho_2=(7.82\pm 0.08)\si[]{\frac{t}{m^3}}$$, für den dritten Stab bei $$\rho_3=(2.783\pm 0.028)\si[]{\frac{t}{m^3}}$$ und für den vierten Stab bei
$$\rho_4=(8.91\pm0.09)\si[]{\frac{t}{m^3}}$$. Anhand der Dichten und Materialfarben kann auf das Metall geschlossen werden. Die Theoretische 
Dichte von Messing liegt bei $\rho_{1,t}=8.4\si[]{\frac{t}{m^3}}$ und weicht damit um 0,35\% vom gemessenen Wert ab ligt jedoch innerhalb des berechneten
Fehlers. Die Dichte von Eisen liegt bei  $\rho_{2,t}=7.874\si[]{\frac{t}{m^3}}$ und weicht somit um 0,69\% vom experimentellen Wert ab leigt aber auch im 
errechneten Fehlerintervall. Die Dichte von Aluminium liegt theoretisch bei $\rho_{3,t}=2.7\si[]{\frac{t}{m^3}}$, die Abweichung vom gemessenen Wert liegt daher 
bei 3,07\% und liegt leicht auserhalb des Fehlerintervalls, allerdings ist die Abweichung zum nächst dichteren Element (Scandium $\rho=2.9$) noch größer
sodass es wahrscheinlich ist das die Abweichung durch eine ungenaue Messung oder dadurch das das Aluminium nicht in reinform vorliegt hervorgerufen wird.
Die Dichte von Kupfer liegt bei $\rho_{4,t}=8.92\si[]{\frac{t}{m^3}}$ und weicht damit um 0.1\% ab liegt allerdings wieder im berechneten Fehler.
Die theoretischen Dichten stammen aus Quelle [2.]. Anschließend werden zuerst die Elastizitätsmodule bestimmt indem der Stab einseitig eingespannt 
wird und an die andere Seite des Stabes eine Masse angehängt wird. Dannach wird der Stab beidseitig eingespannt und eine Masse in der Mitte angehängt.
In beiden fällen wird die absenkung des Stabes gemessen. Die Absenkungen können dann wie in \autoref{sec:nr1} und \autoref{sec:nr2} zu sehen gegen eine
Funktion aufgetragen werden. Aus den Steigungen der errechneten Ausgleichsgraden lassen sich dann die Elastizitätsmodule berechen. Alle berechneten 
Elastizitätsmodule sind in \autoref{tab:ergebnisse2} dargestellt. Die Abweichungen untereinander sowie von den aus Quelle [3.] stammenden Theoriewerten 
wurden über:
\begin{center}
    $Abweichung=\frac{Experimenteller Wert - Theoriewert}{Theoriewert}\times 100$
\end{center}
berechnet und in \autoref{tab:ergebnisse3} dargestellt.
\begin{table}
    \centering
      \caption{In der Tabelle sind alle Abweichungen der errechneten Elastizitätsmodule von den Theoriewerten und die Abweichungen untereinander dargestellt.}
      \label{tab:ergebnisse3}
      \sisetup{table-format=1.2}
      \begin{tabular}{S[table-format=3.2]S S S| S S S S S S S [table-format=3.2]}
        \toprule
        {Nr} & {Material}&{Modul} & { Theoriewert $E_{theo}$ /\GPa}& { Abweichung $E_{theo}$ /\%}& { Abweichung $E_{einseitig}$ /\%} & {Abweichung $E_{links}$ /\%} & {Abweichung $E_{rechts}$ /\%}\\
        \midrule
        1 & {Messing} &{$E_{einseitig}$}  &{$$78...123$$} &{$$0.0$$}&{$$-$$}&{$$404.9$$}&{$$568.6$$}\\
        {$$$$}&{$$$$} &{$E_{links}$}      &{$$$$}         &{$$335.7$$}&{$$404.9$$}&{$$-$$}&{$$40.4$$}\\
        {$$$$}&{$$$$} &{$E_{rechts}$}     &{$$$$}         &{$$471.5$$}&{$$568.6$$}&{$$40.4$$}&{$$-$$}\\
        \midrule
        2 & {Eisen}   &{$E_{einseitig}$}  &{$$196$$}      &{$$10.2$$}&{$$-$$}&{$$304.9$$}&{$$341.2$$}\\
        {$$$$}&{$$$$} &{$E_{links}$}      &{$$$$}         &{$$109.7$$}&{$$304.9$$}&{$$-$$}&{$$8.6$$}\\ 
        {$$$$}&{$$$$} &{$E_{rechts}$}     &{$$$$}         &{$$129.6$$}&{$$341.2$$}&{$$8.6$$}&{$$-$$}\\
        \midrule
        3 & {Aluminium}&{$E_{einseitig}$} &{$$70$$}       &{$$1.4$$}&{$$-$$}&{$$74.2$$}&{$$75.5$$}\\
        {$$$$}&{$$$$}  &{$E_{links}$}     &{$$$$}         &{$$292.9$$}&{$$74.2$$}&{$$-$$}&{$$5.17$$}\\
        {$$$$}&{$$$$}  &{$E_{rechts}$}    &{$$$$}         &{$$314.3$$}&{$$75.5$$}&{$$5.17$$}&{$$-$$}\\
        \midrule
        4 & {Kupfer}  &{$E_{einseitig}$}  &{$$100...130$$}&{$$0.0$$}&{$$-$$}&{$$107.7$$}&{$$138.3$$}\\
        {$$$$}&{$$$$} &{$E_{links}$}      &{$$$$}         &{$$77.7$$}&{$$107.7$$}&{$$-$$}&{$$12.8$$}\\
        {$$$$}&{$$$$} &{$E_{rechts}$}     &{$$$$}         &{$$103.8$$}&{$$138.3$$}&{$$12.8$$}&{$$-$$}\\
        \bottomrule
      \end{tabular}
    \end{table}
Auffällig ist das die durch einseitiges einspannen erhaltenen Werte gut zu den theoretischen Werten passen wärhend die Doppelseitig
eingespannten Stäbe sehr große Abweichungen bringen. Dies kann daran liegen das der Stab bei beidseitiger "Einspannung" genaugenommen
nur einseitig wirklich festgespannt werden konnte während die andere Seite nur auf einem Auflagepunkt auflag. Zudem waren die Stäbe nicht komplett grade.


\section{Literatur}
\label{sec:Literatur}
1. TU-Dortmund, V103 Biegung elastischer Stäbe
2.\hyperlink{https://www.chemie.de/lexikon/}{www.chemie.de} abgerufen am 20.06.2021
3. H. Föll (MaWi 1 Skript),\hyperlink{https://www.tf.uni-kiel.de/matwis/amat/mw1_ge/kap_7/illustr/t7_1_2.html}{www.Uni-Kiel.de} abgerufen am 20.06.2021

\section{Anhang}
\label{sec:anhang}
Auf den nächsten Seiten sind die Originalmesswerte zu finden.
