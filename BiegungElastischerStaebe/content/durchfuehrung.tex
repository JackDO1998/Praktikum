\section{Durchführung}
Für den Versuch wird die Anordnung aus Abbildung () verwendet. Der Stab wird eingespannt und anschließend mit einem Gewicht belasted. Die daraus 
resultierende Biegung aus dem Ausgangszustand lässt sich an den dafür vorgesehenen Messuhren ablesen. Diese müssen jeweils vor Anhängen des Gewichtes 
auf null geeicht werden, um die relative Auslenkung aus dem Ruhezustand ablesen zu können. Der x-Wert des Messpunktes kann an der Anordnung abgelesen werden. \\
\begin{figure}
\centering
\includegraphics[width=12cm, keepaspectratio]{Biegung elastischer Stäbe Versuchsanordnung}
\caption{Anordnung zur Messung der Durchbiegung eines elastischen Stabs}
\label{fig:Aufbau}
\end{figure}
Für die erste Messreihe wird der jeweilge Stab einseitig eingespannt, am einen Ende mit einem festgelegten Gewicht belastet und anschließend an zehn verschiedenen Stellen die Durchbiegung gemessen. Anschließend wird der Stab beidseitig eingespannt und in der Mitte belastet wird. Daraufhin kann mittels zweier Messuhren die Durchbiegung in einem bestimmten Abstand auf beiden Hälften des Stabes gemessen werden. Dies wird für fünf Abstände jeweils auf beiden Seiten und symmetrisch um die Mitte durchgeführt. Zuletzt werden Länge, Gewicht und Durchmesser bzw. Kantenlänge des Stabes bestimmt. Dieses Vorgehen wird für vier Stäbe wiederholt.
