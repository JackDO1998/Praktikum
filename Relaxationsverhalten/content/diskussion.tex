\section{Diskuusion}
In diesem Cersuch wird auf drei verschiedene Arten die Konstante RC berechnet. Mit der
ersten Methode in \autoref{sec:auswertungeins} wird an Wertepaare aus Spannung $U$ und Zeit
$t$ an die Aufladekurve des Kondensators ein Polynom ersten Grades angepasst \autoref{fig:aufladevorgang}, aus dessen Steigung
man auf die Größe von RC schließen kann, diese liegt in dem Fall bei $RC=(5.6\pm0.7)\times 10^{-3}\si[]{s}$. Als nächstes
wird dann in \autoref{sec:auswertungzwei} eine Funktion an Wertepaare aus Kreisfrequenz $\omega$ und Spitze-Spitze-Spannung $U_{SS}$
angepasst. Wie in \autoref{fig:kondensatorspannung} zu sehen ist passt der Fit sehr gut an die Messdaten was dafür spricht das das
Ergebnis von $RC=(1.5046 \pm 0.0986)\times 10^{-3}\si[]{s}$ das genauste ist obwohl es im Vergleich zum Ergebnis aus \autoref{sec:auswertungeins}
weniger als ein Drittel des Betrags groß ist also um etwa 363\% abweicht. Im nächsten Kapitel \autoref{sec:auswertungdrei} wurde dann ein Gesetz an 
Wertepare aus Frequenz $f$ und Phasenverschiebung $\phi$ angepasst. Hier sind im Plot \autoref{fig:verschiebung} deutliche ausreißer zu kleineren 
Verschiebungen hin zu sehen. Auch das Ergebnis von  $RC=(14.37 \pm 5.81)\times10^{-3}\si[]{s}$ weicht stark von den zuvor berechneten ab.
Die Abweichung zur ersten Rechnung liegt bei 256\% und zur Zweiten Messung bei etwa 955\%. Im Weiteren Ist auffällig das Die Fehler sich alle 
nicht überschneiden.  
Die starken Abweichungen können verschiedene Gründe haben, zunächst sind Ablesefehler
in betracht zu ziehen, so wären die beiden Einbrüche in der Phasenverschiebung \autoref{fig:verschiebung}
zu erklären. Im weiteren können ungünstig gewählte Startwerte bei den Fit-Funktionen oder auch systematische
Fehler, wie der Innenwiderstand des Frequenzgenerators, welcher in der Rechnung  als zum Kondensator paralleler
Widerstand berücksichtigt werden müsste. 

\section{Literatur}
1. TU Dortmund, Versuch 353 Relaxationsverhalten des RC-Kreises\\

\section{Anhang}
Auf den nächsten Seiten sind die Originalmesswerte zu finden.
