\section{Zielsetzung}
Bei diesem Experiment wird das Relaxationsverhalten eines RC-Kreises anhand des Aufladevorgangs beobachtet. Außerdem wird die Frequenzabhängigkeit von Phase
und Amplitude eines durch eine Sinusspannung gespeisten RC-Kreises bestimmt.
\section{Theorie}
\aubsection{Der allgemeine Relaxationsvorgang}
Allgemein wird von einem Relaxationsvorgang gesprochen, wenn ein System nach der Auslenkung aus seinem jeweiligen Anfangszustand wieder in selbigen 
zurückkehrt, ohne dabei jedoch oszillatorisches Verhalten an den Tag zu legen. Sowohl die Auf- als auch die Entladung eines Kondensators sind
Beispiele für einen solchen Vorgang. Bei einem Relaxationsvorgang, bei dem sich die betrachtete Größe $A$ asymptotisch seinem Endzustand $A(\infty)$
nähert, ist die Änderungsrate proportional zur Differenz vom Zustand $A(t)$ zum Endzustand:
\begin{equation*}
\frac{dA}{dt}=c(A(t)-A(\infty)
\end{equation*}
was ein allgemeine Differentialgleichung für Relaxationsvorgänge darstellt
Die Lösung dieser Differentialgleichung ergibt sich durch Integration und hat die allgemeine Form:
\begin{equation}
A(t)=A(\infty)+(A(0)+A(\infty))e^ct
\label{eq:allgemeine Relaxationsgleichung}
\end{equation}
mit dem Anfangszustand $A(0)$. Da die Funktion A(t) beschränkt sein soll ergibt sich weiterhin, dass der Parameter $c<0$ gelten muss.
\subsection{Auf- und Entladevorgang des Kondensators}
An einem Kondensator der Kapazität $C$ mit der Ladung $Q$ liegt eine Spannung $U_C$ an, die durch
\begin{equation*}
U_C=\frac{Q}{C}
\end{equation*}
beschrieben wird. Daraus resultiert nach dem Ohmschen Gesetz ein Strom $I=U_C/R$ über den Wiederstand R. Da in der Zeit $dt$ die Ladung 
$Idt$ überfließt ergibt sich für den Strom weiterhin die Darstellung
\begin{equation*}
\frac{dQ}{dt}=-I
\end{equation*}
des Stroms als Ableitung der Ladung nach der Zeit. Mit den vorangegangenen grundlegenden Zusammenhängen für den Kondensator lässt sich daraus die 
eine Differentialgleichung für die Ladung des Kondensators $Q(t)$ in Abhängigkeit von der Zeit herleiten
\begin{equation}
\frac{dQ}{dt}=-\frac{1}{RC}
\end{equation}
welche eine Form der allgemeinen Relaxationsgleichung \ref{allgemeine Relaxationsgleichung} darstellt. Mit Anfangsbedingung $Q(\infty)=0$, die sich ergibt da der 
Kondensator im Endzustand entladen sein soll wird der Entladevorgang durch die Gleichung
\begin{equation}
Q(t)=Q(0)\exp (-t/RC)
\end{equation}
beschreiben. \\
Analog dazu lässt sich für die Anfangsbedingung des Aufladevorgangs, bei dem der Kondensator am Anfang entladen ($Q(0)=0$) sein und am Ende der Aufladung durch die
Spannung $U_0$ die Ladung $Q(\infty)=CU_0$ besitzen soll, die Gleichung 
\begin{equation}
Q(t)=CU_0(1-\exp(-t/RC))
\end{equation}
herleiten. Der Konstante Faktor $RC$ wird als Zeitkonstante bezeichnet und gibt an wie schnell die Auf bzw. Entladung erfolgt. Je größer die Zeitkonstante,
desto langsamer strebt das System seinen jeweiligen Endzustand an.
\subsection{Relaxation unter periodischer Auslenkung}
Der RC-Kreis soll nun durch eine Wechselspannung der Form 
\begin{equation}
U(t)=U_0cos(wt)
\end{equation}
gespeist werden. Dadurch ergibt sich ein Vorgang der wiederum analog zu seinem mechanischen Äquivalent verläuft. Wenn die Frequenz der angelegten Wechselspannung
klein gegenüber der Zeitkonstante ist entspricht die Spannung am Kondensator annähernd der anregenden Spannung. Wenn die Frequenz erhöht wird, verzögert sich die 
Auf bzw. Entladung des Kondensators jedoch gegenüber der anregenden Spannung, woraus sich eine frequenzabhängige Phaseverschiebung $\phi(w)$ ergibt. Daher kann 
die Spannung am Kondensator mit dem Ansatz
\begin{equation}
U_C(t)=A(w)cos(wt+\phi(w))
\end{equation}
mit der Amplitude $A(w)$ beschreiben. Aus vorigen Überlegungen sowie den Kirchhoffschen Regeln lässt sich damit die Gleichung 
\begin{equation*}
U_0\cos(wt)=-AwRC\sin(wt+\phi)+A(w)\cos(wt+\phi)
\end{equation*}
herleiten aus der sich wiederum die Frequenzabhängigkeit von Phase und Amplitude ableiten lassen. Die Phase ergibt sich daraus zu:
\begin{equation}
\phi(w)=\arctan(-wRC)
\end{equation}
Die Phasenverschiebung verschwindet also für $w=0$, und nähert sich bei hohen Frequenzen dem Grenzwert $\phi(\infty)=\pi/2$.
Weiterhin ergibt sich für die Frequenzabhängigkeit der Amplitude der Kondensatorspannung:
\begin{equation}
A(w)=\frac{U_0}{\sqrt{1+w^2R^2C^2}}
\end{equation}
Hieraus lässt sich schließen, dass die Amplitude bei minimaler Frequenz der anregenden Spannung U_0 entspricht ($A(0)=U_0$) während
sie für hohe Frequenzen verschwindet. Aus letzterer Eigenschaft folgt weiterhin auch, dass RC-Kreise als Tiefpässe verwendet werden können,
da sie kleine Frequenzen ungehindert passieren lassen und hohe Frequenzen herausfiltern.
herleiten. Der Konstante Faktor $RC$ wird als Zeitkonstante bezeichnet und gibt an wie schnell die Auf bzw. Entladung erfolgt. Je größer die Zeitkonstante,
desto langsamer strebt das System seinen jeweiligen Endzustand an.

