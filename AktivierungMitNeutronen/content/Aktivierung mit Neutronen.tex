\documentclass{scrartcl}
\usepackage[aux]{rerunfilecheck}
\usepackage[ngerman]{babel}
\usepackage{amsmath}
\usepackage{amssymb}
\usepackage{mathtools}
\usepackage[unicode]{hyperref}
\usepackage{bookmark}
\usepackage{graphicx}
\begin{document}
\section{Zielsetzung}
Im vorliegenden Experiment soll Rhodium durch Neutronen-beschuss radioaktiv aktiviert werden, um anschließend die Zerfallsrate bzw. die Halbwertszeit bestimmen zu können.
\section{Theorie}
\subsection{Grundlagen}
Atomkerne sind nur innerhalb einer bestimmten Anzahl an Neutronen in Relation zu den Protonen stabil. Außerhalb dieses Bereichs zerfällt der Kern unter Aussendung radioaktiver Strahlung und wandelt sich in einen anderen, abhängig vom Ursprungszustand stabilen oder instabilen, Kern um. Die Wahrscheinlichkeit, dass ein solcher Zerfall stattfindet variirt stark abhängig vom beobachteten Nuklid und lässt sich mithilfe der sogenannten Halbwertszeit beschreiben. Diese ist definiert als die Zeitspanne, in der von einer hinreichend großen Anzahl eines bestimmten radioaktiven Nuklids die Hälfte zerfallen ist. Dies bietet sich an, da der Zerfall asymptotisch verläuft und somit nie den Wert $N=0$ erreicht. Vergleichsweise geringe Halbwertszeiten lassen sich am besten messen indem stabile nuklide mit Neutronen beschossen und dadurch instabil werden, ihre Halbwertszeit kann im Anschluss gemessen werden.
\subsection{Kernreaktionen mit Neutronen}
Wenn ein Neutron in einen Kern A eindringt entsteht ein sogenannter Zwischen- oder Compoundkern $A*$, dessen Energie um die Gesamtenergie des absorbierten Neutrons größer als die des Ausgangskerns A ist.  Diese zusätzliche Energie verteilt sich über die Nukleonen und erhöht deren Energiezustand. Dies führt nach ca. $10^{-16}$ s zur Emmission eines $\gamma$-Quants, sodass der Kern unter folgender Reaktion in seinen Ursprungszustand zurückkehrt.
\begin{equation*}
^m_zA + ^1_0n \implies ^{m+1}_zA* \implies ^{m+1}_z A+ \gamma
\end{equation*}
Aufgrund des zusätzlichen Neutrons ist dieser Kern instabil und zerfällt unter Emission eines Elektrons zu einem stabilen Kern
\begin{equation*}
^{m+1}_z A \implies  ^{m+1}_{z+1} C + \beta^- + E_{kin} + v_e
\end{equation*}
($v_e = Antineutrino$).
\subsection{Wirkungsquerschnitt}
Der Wirkungsquerschnitt  $\sigma$ ist eine imaginäre Fläche, die die Wahrscheinlichkeit für das Einfangen eines Neutrons beschreibt. Sie wird so gewählt, dass jedes Neutron, welches auf diese Fläche trifft, vom Nuklid eingefangen werden würde.
Die Wahrscheinlichkeit und damit der Wirkungsquerschnitt hängt dabei auch von der Geschwindigkeit des Neutrons ab. Ausgehend davon, dass eine Absorption immer dann Eintritt, wenn die Neutronenenergie der Differenz zweier Energieniveaus von $A*$ entspricht, lässt sich der Wirkungsquerschnitt gemäß
\begin{equation*}
\sigma(E)=\sigma_0 \sqrt{\frac{E_{ri}}{E}}\frac{c}{(E-E_{ri})^2+c}
\end{equation*}
mit den Konstanten $\sigma_0$, $c$ und den Energieniveaus $E_{ri}$ beschreiben. Daraus folgt, dass der Wirkungsquerschnitt umgekehrt proportional zur Geschwindigkeit ist, was sich darauf zurückführen lässt, dass sich ein langsames Elektron länger im Wirkungsbereich des Nuklids befindet.
\subsection{Erzeugung von Neutronen}
Davon ausgehend bieten sich für experimentelle Untersuchungen niederenergetische Neutronen mit einem dementsprechend hohen Wirkungsquerschnitt an. Diese lassen sich durch Beschuss von Berillium Kernen mit $\alpha$-Strahlung gewinnen
\begin{equation*}
^9_4Be + ^4_2\alpha \implies ^{12}_6C + ^1_0n
\end{equation*}
Die so erzegten Neutronen werden gebremst, indem sie durch mehrere Materieschichten geleitet werden und dort einen Anteil ihrer Energie durch elastische Stöße an die Materie abgeben. Da sich für den maximalen Energieübertrag bei einem solchen Stoß möglichst ähnliche Massen am besten eignen, wird als Stoßpartner Paraffin verwendet. Aus diesem Prozess resultieren sogenannte thermische Neutronen mit der benötigten vergleichsweise geringen mittleren Geschwindigkeit von ca.$2.2$ km/s.
\subsection{Zerfall radioaktiver Isotope}
Bestimmte Isotope lassen sich wie in Kapitel $2.2$ beschrieben durch diese Neutronen Aktivieren und stabilisieren sich unter $\beta^-$-Zerfall mit einer Halbwertszeit von einigen Sekunden bis zu einer Stunde. Die Zahl der verbliebenen nicht zerfallenen Kerne $N$ als Funktion der Zeit lässt sich mithilfe des Zerfallsgesetzes
\begin{equation}
N(t)=N_0e^{-\lambda t}
\end{equation}
beschreiben. Dabei ist $\lambda$ die sogenannte Zerfallkonstante und $N_0$ der Anfangswert, also die Zahl der Kerne zum Zeitpunkt $t=0$. Die Zerfallskonstante hängt von der Wahrscheinlichkeit für den Zerfall ab und steht in direkter Relation zur Halbwertszeit $T$. Diese ergibt sich aus $N(t=T)=N_0/2$ zu
\begin{equation}
T=\frac{ln(2)}{\lambda}
\end{equation}
Umgekehrt lässt sich durch Messung von $T_{1/2}$ die Zerfallkonstante
\begin{equation*}
\lambda=\frac{ln(2)}{T}
\end{equation*}
bestimmen. Da die Zahl der nicht zerfallenen Kerne $N(t)$ ein äußerst schwierig zu erhebender Wert ist, bietet es sich an  stattdessen die Zahl der in einem fest definierten Zeitintervall stattfindenden Zerfälle mit einem Zählrohr zu bestimmen . Dieser Wert ergibt sich aus dem Zerfallsgesetz zu
\begin{gather}
N_{\Delta t}(t)=N(t)-N(t + \Delta t)=N_0e^{-\lambda t}-N_0e^{-\lambda (t+\Delta t)} \\
N_{\Delta t}(t)=N_0e^{-\lambda t}(1-e^{-\lambda \Delta t} ) \\
\iff ln(N_{\Delta t}(t))=-\lambda t + ln(N_0(1-e^{-\lambda \Delta t}))
\end{gather}
Aus der letzten Gleichung kann mithilfe einer linearen Ausgleichsrechnug die Zerfallskonstante bestimmt werden, da $ln(N_0(1-e^{\lambda \Delta t}))$ nur von konstanten Faktoren abhängt. Für diesen Ansatz ist es wichtig eine passende Messzeit $\Delta$ t zu wählen, da bei einem zu kleinen Intervall ein großer Messfehlerauftritt, wogegen ein großes Intervall eine scheinbare $\Delta$t Abhängigkeit von $\lambda$ vortäuscht und somit zu einem systematischen Messfehler führt. \\
In diesem Experiment wurden Messwerte für zwei verschiedene Isotope aufgenommen. Zum einen wurde Vanadium verwendet, welches gemäß folgender Gleichung
\begin{equation}
^{51}_{23}V + ^1_0n \implies ^{52}_{23}V \implies ^{52}_{24}Cr + \beta^- + v_e
\end{equation}
nach dem Neutronen-beschuss unter Emission von $\beta^-$-Strahlung zu Chrom zerfällt. Des weiteren wurden Messdaten für Rhodium erhoben, für welches zwei verschiedene Zerfälle mit unterschiedlichen Wahrscheinlichkeiten eintreten
\begin{equation*}
^{103}_{45}Rh+^1_0n=
\begin{cases}
{\stackrel{10\%}{\implies}} &^{104i}_{45}Rh \implies ^{104}_{45}Rh+ \gamma \implies ^{104}_{46}Pd+\beta^-+v_e \\
{\stackrel{90\%}{\implies}} &^{104}_{45}Rh \implies ^{104}_{46}Pd+\beta^-+v_e \\
\end{cases}
\end{equation*}
Da das Geiger-Müller-Zählrohr sowohl $\beta^-$- als auch $\gamma$-Strahlung detektieren kann entspricht die Messrate der Summe beider Zerfälle, die simultan ablaufen. Die Zerfälle besitzen unterschiedliche Halbwertszeiten, sodass nach einer bestimmten Zeitspanne $t*$ der kurzlebigere Verfall vernachlässigbar klein wird und nur noch der langlebigere Zerfall verbleibt. Für diesen kann (für $t>t*$) die Zerfallskonstante bestimmt werden. Anschließend kann die aus dieser Konstante berechnete Zahl an Zerfällen für $t<t*$ von der Gesamtaktivität abgezogen werden um den anderen Zerfall zu erhalten.
\subsection{Nulleffekt}
Aufgrund diverser natürlicher Phänomene existiert ein Grundwert für Strahlung der als Nulleffekt bezeichnet wird. Um exakte Messwerte zu erhalten muss zunächst der Nulleffekt gemessen und anschließend von späteren Messungen abgezogen werden.
\section{Durchführung}
\subsection{Aufbau}
Für den Versuch wurde eine Anordnung entsprechend der Darstellung verwendet. Das Zählrohr wird hinter einer Blei-Abschirmung platziert um den Einfluss des Nulleffektes zu minimieren. Die vom Zählrohr registrierten Impulse werden an einen Zähler weitergeleitet, an welchem das gewünschte Messintervall $\Delta t$ eingestellt werden kann. Er besitzt zwei Anzeigevorrichtungen, zwischen denen nach $\Delta t$ umgeschaltet wird, sodass zu jedem Zeitpunkt eine der Anzeigen zählt und die jeweils andere das Ergebnis des vorrangegangenen Zeitintervalls anzeigt.
\subsection{Messungen}
Zunächst wurde der Nulleffekt gemessen, wobei ein großes Zeitintervall von $\Delta t=300$ s verwendet wurde, um den Messfehler zu minimieren. \\
Anschließend wurden Messwerte für das Vanadium direkt nach Aktivierung der Probe erhoben, dabei wurde als Zeitintervall $\Delta t=30$ s gewählt. Eine analoge Messung wurde für Rhodium durchgeführt, jedoch mit einer Messzeit von $\Delta t=15$ s.  
\end{document}