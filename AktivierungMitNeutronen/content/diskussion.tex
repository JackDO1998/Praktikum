\section{Diskussion}
\label{sec:diskussion}
In diesem Versuch sollten die Zerfallskonstenten $\lambda$ des Zerfalls von $^{51}_{23}\mathrm{V}$ und den 
beiden Zerfällen von $^{103}_{45}\mathrm{Rh}$ ermittelt werden. Dazu wurde an die mittels Geiger-Müller-Zählrohr
ermittelten Zerfallszahlen das Zerfallsgesetz angepasst. Auf diese Weise wurde zunächst die Zerfallskonstante
von Vanadium bestimmt, sie beläuft sich auf: $0.003\frac{1}{s}$ der Literaturwert liegt bei $0.003086\frac{1}{s}$
und weicht damit nur um etwa 2,8\% von der errechneten Größe ab. Die Zerfallskonstante für den langsamen Rhodium-Zerfall
liegt der Berechnung nach bei $\lambda_{Rh1} =0.004\pm 0.001\frac{1}{s}$ für den schnelleren bei $\lambda_{Rh2} =0.015\frac{1}{s}$.
Es konnte in der Literatur nur ein Wert für die Zerfallskonstante ermittlet werden sie liegt demnach bei 
$0.01639\frac{1}{s}$, was einer Abweichung von mindesten 9,3\% entspricht. Die sehr kleine Abweichung beim Vanadium
kann leicht durch Verfälschungen durch die Untergrundstrahlung erklärt werden welche zwar im Vorhinein 
ausgemessen wurde jedoch nur als gemittelte Größe von den späteren Messungen der Kernreaktion an sich abgeszogen wurde.
Es können also keine genauen Aussagen zur Untergrundstrahlung während der Messung getroffen werden.
Die selben Gründe können für den großen Fehler bei der Messung der Rhodium Zerfallskonstanten aufgeführt werden.
Hier ist allerdings auch zu beachten das der schnelle Zerfall nach kurzer Zeit weitgehend, aber nicht vollständig 
abgeschlossen ist und so das Ergebnis für den langsamen Zerfall verfälscht. Da das Ergebnis für den schnellen Zerfall
allerdings vom Ergebnis für den langsamen Zerfall abhängt führt dieser Umstand auch zu einer Verfälschung 
des Messergebnisses. Eine weitere Problematik besteht darin, das es beim langsamen Zerfall bedeutend weniger 
Zerfälle gibt sodas viel davon im Untergrund verschwindet und zu großen Ungenauigkeiten führt. Als Resultat kann
gesagt werden das die Messung der Vanadium-Zerfallskonstante sehr gut funktioniert hat, während die der 
Rhodium-Zerfallskonstante nur ausreichte um die Größenordung festzustellen. 



\section{Literatur}
\label{sec:literatur}
1. TU-Dortmund, V702 Aktivierung mit Neutronen\\
2. http://www.periodensystem-online.de/ (01.02.2021) - Die Zerfallskonstanten

\section{Anhang}
\label{sec:anhang}
Auf den folgenden Seiten finden sich die Originalmesswerte.