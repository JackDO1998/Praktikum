\section{Diskussion}
\label{sec:diskussion}
In diesem Versuch sollten die Halbwertszeiten $T_{1/2}$ des Zerfalls von $^{51}_{23}\mathrm{V}$ und den 
beiden Zerfällen von $^{103}_{45}\mathrm{Rh}$ ermittelt werden. Dazu wurde an die mittels Geiger-Müller-Zählrohr
ermittelten Zerfallszahlen das Zerfallsgesetz angepasst. Auf diese Weise wurde zunächst die Zerfallskonstante
von Vanadium bestimmt, sie beläuft sich auf: $(0.00342\pm 0.00012)\frac{1}{s}$ der Literaturwert liegt bei $0.003086\frac{1}{s}$
und weicht damit nur um etwa 10,8\% von der errechneten Größe ab. Da die Messwerte ab etwa 660 Sekunden
in der Größenordung ihres eigenen Fehlers liegen wurde die Zerfallskonstante nochmal bis zur 660. Sekunde
berechnet diese liegt dann bei $(0.00342\pm 0.00018)\frac{1}{s}$, weicht also ebenfalls um circa 10,8\% 
vom Literaturwert ab hat allerdings einen geringfügig größeren Fehler.
Die Zerfallskonstante für den langsamen Rhodium-Zerfall
liegt der Berechnung nach bei $\lambda_{Rh1} =(0.004\pm 0.001)\frac{1}{s}$ für den schnelleren bei $\lambda_{Rh2} =0.015\frac{1}{s}$.
Es konnte in der Literatur nur ein Wert für die Zerfallskonstante ermittlet werden sie liegt demnach bei 
$0.01639\frac{1}{s}$, was einer Abweichung von mindesten 9,3\% entspricht. Als nächstes konnten aus den 
Zerfallskonstanten die Halbwertszeiten bestimmt werden, diese liegt für
Vanadium bei $T_{1/2}=(203\pm 7)\si[]{s}$, bzw. für die Messung bis $660\si[]{s}$ bei $(203\pm 11)\si[]{s}$ der zugehörige Literaturwert liegt $T_{1/2L}=224,6\si[]{s}$ und weicht
damit je nur um etwa 10,3\% ab. Für den schnellen Rhodiumzerfall liegt die errechnete Halbwertszeit bei 
$T_{1/s}=(46.2098\pm0)\si[]{s}$, die Halbwertszeit in der Literatur liegt bei $T_{1/2L}=42.3\si[]{s}$ und 
weicht damit um etwa 9\% ab. Für den zweiten Zerall von Rhodium konnte kein Literaturwert zum Vergleich gefunden werden.
Die sehr kleine Abweichung beim Vanadium
kann leicht durch Verfälschungen durch die Untergrundstrahlung erklärt werden welche zwar im Vorhinein 
ausgemessen wurde jedoch nur als gemittelte Größe von den späteren Messungen der Kernreaktion an sich abgeszogen wurde.
Es können also keine genauen Aussagen zur Untergrundstrahlung während der Messung getroffen werden.
Die selben Gründe können für den großen Fehler bei der Messung der Rhodium Zerfallskonstanten aufgeführt werden.
Hier ist allerdings auch zu beachten das der schnelle Zerfall nach kurzer Zeit weitgehend, aber nicht vollständig 
abgeschlossen ist und so das Ergebnis für den langsamen Zerfall verfälscht. Da das Ergebnis für den schnellen Zerfall
allerdings vom Ergebnis für den langsamen Zerfall abhängt führt dieser Umstand auch zu einer Verfälschung 
des Messergebnisses. Eine weitere Problematik besteht darin, das es beim langsamen Zerfall bedeutend weniger 
Zerfälle gibt sodas viel davon im Untergrund verschwindet und zu großen Ungenauigkeiten führt. Weiter Fehler
können durch Ablesefehler und nicht mitgezählte Zerfälle beim umschalten der Zählvorrichtung am Messgerät
entstehen sowie durch den Umstand das die Probe nach ihrer Erzeugung erst zum Messgerät gebracht werden muss und während
dieser Zeit natürlich schon teilweise zerfällt. Zudem wurde der Zeitpunkt ab dem der schnellere Zerfall abgeschlossen
ist völlig willkürlich gewählt, wenn dieser also schlecht gewählt wurde verfälscht auch das das Ergebnis. Als Resultat kann
gesagt werden das die Messung der Vanadium-Zerfallskonstante sehr gut funktioniert hat, während die der 
Rhodium-Zerfallskonstante nur ausreichte um die Größenordung festzustellen. 



\section{Literatur}
\label{sec:literatur}
1. TU-Dortmund, V702 Aktivierung mit Neutronen\\
2. http://www.periodensystem-online.de/ (01.02.2021) - Die Zerfallskonstanten

\section{Anhang}
\label{sec:anhang}
Auf den folgenden Seiten finden sich die Originalmesswerte.