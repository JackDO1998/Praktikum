\section{Diskussion}
In diesem Versuch wurde zunächst in \autoref{sec:reflexionsgestz} das reflexionsgestz
überprüft. Hier entsteht, im Rahmen der eingeschränkten Messgenauigkeit,
bei keienem aufgenommenen Messwert eine Abweichung von der Erwartung. Die Messgenauigkeit ist eingeschränkt
da der Lasepunkt auf dem Schirm eine Ausdenung besitzt und die Skala nur gradweise und nicht direkt auf dem Schirm
sondern nur darunter aufgetragen ist.
Im nächsten Schritt, \autoref{sec:brechungsgesetz}, wird dann der Brechungsindex von Plexiglas
bestimmt. Der hier gemessene Wert liegt bei $n_{plex}=1,474\pm 0,014$ und weicht somit um etwa 1,08\% 
von dem in \autoref{sec:vorbereitung1} nachgeschlagenen Wert ab und liegt so knapp außerhalb des Fehlerintervalls.
Ebenfalls wurde in diesem Kapitel die Lichtgeschwindigkeit in Plexiglas bestimmt, sie liegt bei 
$v_2=(2.034\pm 0.019)\times 10^8\si[]{m/s}$ und weicht somit ebenfalls um etwa 1\% vom Theoriewert
$v_{theo}=\SI[]{201202991}[]{m/s}$ ab. Im Anschluss daran wird in \autoref{sec:platten} der Strahlversatz an planparallelen platten
auf zwei verschiedene Weisen berechent. Die berechneten werte liegen bei $s_1=1.68\pm 0.30\si[]{cm}$ 
und $s_2=1,67\pm 0,28\si[]{cm}$. Sie unterscheiden sich also nur um etwa 0,6\% und liegen gegnseitig 
im Fehlerintervall. In \autoref{sec:prisma} werden dann die Wellenlängen abhängigen Brechungswinkel welche 
durch ein Prisma erszeugt werden gemessen sie liegen für den grünen Laser bei $\delta_g=41,6\pm 1,7\si[]{°}$ 
und für den roten laser bei $\delta_r=40.4\pm 1.6\si[]{°}$. Hier leigen keine Theoriewerte zum Vergleich vor.
Im letzten Kapitel \autoref{sec:beugung} werden dann die Wellenlängen der benuzten Laser zu 
$\lambda_{rot}=644,0\pm 6\si[]{nm}$ und $\lambda_{grün}=532,7\pm 3.5\si[]{nm}$ bestimmt. Es liegen auch hier 
keine Werte zum Vergleich vor, jedoch sind die berechneten Wellenlängen in einem für die Lichtfarben typischen
Bereich. Fehler können in diesen Versuchen vorallem durch ungenaue oder schlecht ablesbare Skalen entstehen. 
Die gemessenen Werte zeigen jedoch durchweg eine geringe Abweichung von der Erwartung. Die prozentualen Abweichungen
werden über \autoref{eq:prozentuale} berechenet.
\section{Literatur}
1. TU-Dortmund, V204: Wärmeleitung von Metallen\\
2. Brechungsindexdatenbank,\hyperlink{https://www.filmetrics.de/refractive-index-database}{www.filmetrics.de} abgerufen am 26.06.2021\\
\section{Anhang}
Auf der folgenden Seite befinden sich ein Scan der Orginalwerte.