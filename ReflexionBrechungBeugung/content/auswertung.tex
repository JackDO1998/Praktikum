\section{Auswertung}
In diesem Kapitel sollen die aufgenommenen Messwerte Ausgewertet und verrechnet werden.
\subsection{Vorbereitungsaufgaben}
Hier sind die Ergebnisse der Vorbereitungsaufgaben dargestellt.
\subsubsection{Brechungsindizes}
\label{sec:vorbereitung1}
In Quelle 2 können folgende Brechungsindizes nachgeschlagen werden.
Luft: $n_{Luft}=1$,\\
Wasser: $n_{Wasser}=1,333$,\\
Kronglas: $n_{Kron}=1,5$,\\
Plexiglas: $n_{Plex}=1,49$,\\
Diamant: $n_{Dia}=2,42$.
\subsubsection{Gitterkonstanten}
\label{sec:vorberitung2}
Die Gitterkonstanten sind der Kehrwert der Anzahl der Gitterlinien pro mm. Das führt zu den Gitterkonstanten:
\begin{center}
  $d_1=\SI[]{0,01}[]{1/mm}$ für 100L/mm,\\
  $d_2=\SI[]{0,00\bar{3}}[]{1/mm}$ für 300L/mm und\\
  $d_3=\SI[]{0,001\bar{6}}[]{1/mm}$ für 600L/mm.\\
\end{center}
\subsection{Überprüfung des Reflexionsgesetzes}
\label{sec:reflexionsgestz}
Nach dem Reflexionsgesetz ist der Einfallswinkel gleich dem Ausfallswinkel($\alpha_1=\alpha_2$). Es wird
aus verschiedenen definierten Winkeln $\alpha_1$ ein Laserstrahl auf einen Spiegel gerichtet und der 
Reflexionswinkel $\alpha_2$ von einem Schirm abgelesen. Die theoretischen und gemessenen Winkel
sowie die Abweichung sind in \autoref{tab:reflexionsgestz} dargestellt.

\begin{table}
  \centering
    \caption{Einfalls- und Ausfallswinkel eines auf einen Spiegel gerichteten Laserstrahls.}
    \label{tab:ergebnisse}
    \sisetup{table-format=1.2}
    \begin{tabular}{S[table-format=3.2] S S S S S [table-format=3.2]}
      \toprule
      { $\alpha_1$ / °}&{ $\alpha_2$ / °}&{ $\alpha_{2,theo}$ / °}&{ Abweichung / \%}\\
      \midrule
      {$$20$$}  &{$$20\pm 1$$}   &{$$20$$}  &{$$0,0$$} \\
      {$$30$$}  &{$$30\pm 1$$}   &{$$30$$}  &{$$0,0$$}  \\
      {$$35$$}  &{$$35\pm 1$$}   &{$$35$$}  &{$$0,0$$}  \\
      {$$40$$}  &{$$40\pm 1$$}   &{$$40$$}  &{$$0,0$$} \\
      {$$45$$}  &{$$45\pm 1$$}   &{$$45$$}  &{$$0,0$$}  \\
      {$$50$$}  &{$$50\pm 1$$}   &{$$50$$}  &{$$0,0$$}  \\
      {$$60$$}  &{$$60\pm 1$$}   &{$$60$$}  &{$$0,0$$}  \\
      \bottomrule
    \end{tabular}
  \end{table}
Da der Lasepunkt auf dem Schirm eine Ausdehnung von dem Equivalent eines Grades hat und die 
Skala für Einfalls- und Ausfallswinkel ebenfalls nur gradweise eingeteilt ist können 
die Winkel auch nur auf höchstens $\pm \SI[]{1}[]{°}$ genau bestimmt werden.

\subsection{Brechungsgesetz}
\label{sec:brechungsgesetz}
Um den Brechungsindex zu bestimmen wird der Brechungswinkel bestimmt. Der Brechungsindex
kann dann leicht über den Zusammenhang:
\begin{center}
  $n_2=\frac{sin \alpha}{sin \beta}$\\
\end{center}
bestimmt werden. Es werden die Werte für $n_2$ einzeln bestimmt, anschließend über \autoref{eq:Mittelwert}
gemittelt und in \autoref{tab:brechungsgesetz} dargestellt. Die Fehler ergeben sich zunächst über \autoref{eq:gaussfehler} und dann für den Mittelwert über
\autoref{eq:mittelwertfehler}
\begin{table}
  \centering
    \caption{Brechung eines Laserstrahls im Medium.}
    \label{tab:brechungsgesetz}
    \sisetup{table-format=1.2}
    \begin{tabular}{S[table-format=3.2] S S S S S [table-format=3.2]}
      \toprule
      { $\alpha_1$ / °}&{ $\beta$ / °}&{ $n_2$ }\\
      \midrule
      {$$20$$}  &{$$14,0\pm 0,5$$}   &{$$1,413\pm 0,084$$}    \\
      {$$30$$}  &{$$20,5\pm 0,5$$}   &{$$1,427\pm 0,055$$}    \\
      {$$35$$}  &{$$22,5\pm 0,5$$}   &{$$1,499\pm 0,049$$}    \\
      {$$40$$}  &{$$25,5\pm 0,5$$}   &{$$1,493\pm 0,041$$}    \\
      {$$45$$}  &{$$28,0\pm 0,5$$}   &{$$1,506\pm 0,036$$}    \\
      {$$50$$}  &{$$31,0\pm 0,5$$}   &{$$1,487\pm 0,031$$}    \\
      {$$60$$}  &{$$35,5\pm 0,5$$}   &{$$1,491\pm 0,024$$}    \\
      \midrule
      {$$\diameter$$}&{$$$$}&{$$1,474\pm 0,014$$}\\
      \bottomrule
    \end{tabular}
  \end{table}
  Mit dem berechneten Wert für $n$ lässt sich über die folgende Beziehung leicht die Lichtgeschwindigkeit
  $v_2$ im Medium berechnen.
 \begin{center}
   $v_2=\frac{v_1}{n_2}$, $v_1=c=29979\times 10^8 \si[]{m/s}$
 \end{center}
 Diese liegt demnach bei $v_2=(2.034\pm 0.019)\times 10^8\si[]{m/s}$


 \subsection{Planparallele Platten}
 \label{sec:platten}
 Wenn Licht durch zwei planparallele Schichtgrenzen hindurchfällt erfährt es einen Strahlversatz
 $s$. Dieser berechnet sich über:
 \begin{center}
   $s=d\frac{sin(\alpha-\beta)}{cos\beta}$.\\
 \end{center}
 Es wird einerseits der Einfallswinkel $\alpha_1$ und der Brechungswinkel $\beta_{gemessen}$ gemessen und
 andererseits wird der Winkel $\beta_{berechent}$ aus dem Einfallswinkel und dem Brechungsindex aus 
 \autoref{sec:brechungsgesetz} berechenet. Für beide wird dann der Strahlversatz berechnet und in \autoref{tab:versatz}
 dargestellt. Die Fehler ergeben sich über die Gaußsche-Fehlerfortpflanzung \autoref{eq:gaussfehler}. Die am Ende
 berechneten Mittelwerte ergeben sich mit ihrem Fehler über \autoref{eq:Mittelwert} und \autoref{eq:mittelwertfehler}.
 \begin{table}
  \centering
    \caption{Brechung eines Laserstrahls im Medium.}
    \label{tab:versatz}
    \sisetup{table-format=1.2}
    \begin{tabular}{S[table-format=3.2] S S S S S [table-format=3.2]}
      \toprule
      { $\alpha_1$ / °}&{ $\beta_{gemessen}$ / °}&{ $s_1$ /cm }&{ $\beta_{berechnet}$ / °}&{ $s_2$ /cm }\\
      \midrule
      {$$20$$}  &{$$14,0\pm 0,5$$}&{$$0,630\pm 0,116$$}&{$$13,417\pm 0,668$$}   &{$$0,689\pm 0,040$$}    \\
      {$$30$$}  &{$$20,5\pm 0,5$$}&{$$1,031\pm 0,119$$}&{$$19,828\pm 0,655$$}   &{$$1,098\pm 0,049$$}    \\
      {$$35$$}  &{$$22,5\pm 0,5$$}&{$$1,370\pm 0,118$$}&{$$22,899\pm 0,646$$}   &{$$1,331\pm 0,054$$}    \\
      {$$40$$}  &{$$25,5\pm 0,5$$}&{$$1,623\pm 0,120$$}&{$$25,854\pm 0,635$$}   &{$$1,589\pm 0,060$$}    \\
      {$$45$$}  &{$$28,0\pm 0,5$$}&{$$1,937\pm 0,120$$}&{$$28,666\pm 0,623$$}   &{$$1,875\pm 0,067$$}    \\
      {$$50$$}  &{$$31,0\pm 0,5$$}&{$$2,222\pm 0,121$$}&{$$31,312\pm 0,609$$}   &{$$2,194\pm 0,074$$}    \\
      {$$60$$}  &{$$35,5\pm 0,5$$}&{$$2,980\pm 0,120$$}&{$$35,981\pm 0,577$$}   &{$$2,943\pm 0,088$$}    \\
      \midrule
      {$$\diameter$$}&{$$$$}&{$$1.68\pm 0.30$$}&{$$$$}&{$$1,67\pm 0,28$$}\\
      \bottomrule
    \end{tabular}
  \end{table}
  Die berechenten Mittelwerte unterscheiden sich kaum voneinander und liegen im gegenseitigen Fehlerintervall.
  
  
  \subsection{Das Prisma}
  \label{sec:prisma}
  In einem Prisma wird Licht in abhängigkeit von seiner Wellenlänge $\lambda$ gebrochen. In diesem Versuch 
  trifft ein Laserstrahl in unterschiedlichen Winkeln $\alpha_1$ auf das Prisma und tritt im Winkel 
  $\alpha_2$ wieder aus. Die anderen Winkel ergeben sich dann über die Beziehungen:

  \begin{center}
    $\beta_2=arcsin(\frac{sin\alpha}{n_{kron}})$,\\
    $\beta_1=\gamma-\beta_2$ und\\
    $\delta=(\alpha_1+\alpha_2)-(\beta_1+\beta_2)$.\\
  \end{center}
  Mit $n_{kron}$ dem Bechungsindex von Kronglas aus \autoref{sec:vorbereitung1}
  Für den grünen Laserstrahl sind die Ergebnisse in \autoref{tab:prismagrun} und für den roten in 
  \autoref{tab:prismarot} dargestellt.
  Die Messfehler ergeben sich hierbei über \autoref{eq:gaussfehler} und der am Ende berechnetet Mittelwert
  ergibt sich mit dem zugehörigen Fehler über \autoref{eq:Mittelwert} und \autoref{eq:mittelwertfehler}.
  \begin{table}
    \centering
      \caption{Dispersion eines grünen Laserstrahls im Medium.}
      \label{tab:prismagrun}
      \sisetup{table-format=1.2}
      \begin{tabular}{S[table-format=3.2] S S S S S [table-format=3.2]}
        \toprule
        { $\alpha_{1,grün}$ / °}&{ $\alpha_{2,grün}$ / °}&{ $\beta_{1,grün}$ / ° }&{ $\beta_{2,grün}$ / °}&{ $\delta$ / ° }\\
        \midrule
        {$$30$$}  &{$$81\pm 1$$}& {$$17,929\pm 0,512$$}& {$$42,071\pm 0,513$$}   &{$$51.0\pm 1,414$$}    \\
        {$$35$$}  &{$$68\pm 1$$}& {$$21,023\pm 0,549$$}& {$$38,977\pm 0,549$$}   &{$$43,0\pm 1,414$$}    \\
        {$$40$$}  &{$$61\pm 1$$}& {$$23,605\pm 0,573$$}& {$$36,395\pm 0,573$$}   &{$$41,0\pm 1,414$$}    \\
        {$$55$$}  &{$$43\pm 1$$}& {$$32,440\pm 0,628$$}& {$$27,559\pm 0,628$$}   &{$$38,0\pm 1,414$$}    \\
        {$$60$$}  &{$$38\pm 1$$}& {$$35,312\pm 0,640$$}& {$$24,688\pm 0,639$$}   &{$$38,0\pm 1,414$$}    \\
        {$$65$$}  &{$$34\pm 1$$}& {$$37,706\pm 0,648$$}& {$$22,294\pm 0,672$$}   &{$$39,0\pm 1,414$$}    \\
        {$$70$$}  &{$$31\pm 1$$}& {$$39,549\pm 0,653$$}& {$$20,451\pm 0,653$$}   &{$$41,0\pm 1,414$$}    \\
        \midrule
        {$$\diameter$$}&{$$$$}&{$$$$}&{$$$$}&{$$41,6\pm 1,7$$}\\
        \bottomrule
     \end{tabular}
    \end{table}
    
    

    \begin{table}
     \centering
        \caption{Dispersion eines roten Laserstrahls im Prisma.}
        \label{tab:prismarot}
        \sisetup{table-format=1.2}
        \begin{tabular}{S[table-format=3.2] S S S S S [table-format=3.2]}
          \toprule
          { $\alpha_{1,rot}$ / °}&{ $\alpha_{2,rot}$ / °}&{ $\beta_{1,rot}$ / ° }&{ $\beta_{2,rot}$ / °}&{ $\delta$ / ° }\\
          \midrule
          {$$30$$}  &{$$79\pm 1$$}&{$$18,245\pm 0,517$$}&{$$41,755\pm 0,517$$}   &{$$49,0\pm 1,414$$}    \\
          {$$35$$}  &{$$67\pm 1$$}&{$$21,356\pm 0,552$$}&{$$38,643\pm 0,552$$}   &{$$42,0\pm 1,414$$}    \\
          {$$40$$}  &{$$60\pm 1$$}&{$$24,019\pm 0,577$$}&{$$35,981\pm 0,577$$}   &{$$40,0\pm 1,414$$}    \\
          {$$55$$}  &{$$42\pm 1$$}&{$$33,003\pm 0,630$$}&{$$26,997\pm 0,630$$}   &{$$37,0\pm 1,414$$}    \\
          {$$60$$}  &{$$37\pm 1$$}&{$$35,903\pm 0,642$$}&{$$24,097\pm 0,642$$}   &{$$37,0\pm 1,414$$}    \\
          {$$65$$}  &{$$33\pm 1$$}&{$$38,316\pm 0,650$$}&{$$21,684\pm 0,649$$}   &{$$38,0\pm 1,414$$}    \\
          {$$70$$}  &{$$30\pm 1$$}&{$$40,172\pm 0,655$$}&{$$19,828\pm 0,655$$}   &{$$40,0\pm 1,414$$}    \\
          \midrule
          {$$\diameter$$}&{$$$$}&{$$$$}&{$$$$}&{$$40.4\pm 1.6$$}\\
          \bottomrule
      \end{tabular}
    \end{table}

  \subsection{Beugung am Gitter}    
  \label{sec:beugung}
  Um die Wellenlänge der verwendeten Laser zu messen werden drei optische Beugungsgitter in den
  Strahlverlauf gestellt und die Beugungswinkel gemessen. Aus diesen lässt sich über die Beziehung:
   \begin{center}
     $\lambda=\frac{d sin\alpha}{k}$
   \end{center}   
mir $d$ den in \autoref{sec:vorbereitung2} berechneten Gitterkonstanten und $k$ der Beugungsordnung,
die Wellenlänge berechnen. Der Fehler pflanzt sich dabei über \autoref{eq:gaussfehler} fort. Die berechneten
$\lambda$ sind in \autoref{tab:gitter} dargestellt und über \autoref{eq:Mittelwert} gemittelt worden.
Der Mittelwertfehler ergibt sich dabei mit \autoref{eq:mittelwertfehler}.
\begin{table}
  \centering
     \caption{Beugung des roten und grünen Laserstrahls an verschiedenen Gittern.}
     \label{tab:gitter}
     \sisetup{table-format=1.2}
     \begin{tabular}{S[table-format=3.2] S| S S S S [table-format=3.2]}
       \toprule
       { $d$ / 1/mm}&{$k$}&{ $\alpha_{rot}$ / °}&{$\lambda_{rot}$ / $\mu m$}&{ $\alpha_{grün}$ / ° }&{$\lambda_{rot}$ / $\mu m$}\\
       \midrule
       {$$0,01$$}&{$$1$$}&{$$4\pm 1$$} &{$$0,697\pm 0,174$$}&{$$3\pm 1$$} &{$$0,523\pm 0,174$$}\\
       {$$$$}&{$$2$$}&{$$7\pm 1$$} &{$$0,609\pm 0,087$$}&{$$6\pm 1$$} &{$$0,523\pm 0,087$$}\\
       {$$$$}&{$$3$$}&{$$11\pm 1$$}&{$$0,636\pm 0,057$$}&{$$9\pm 1$$} &{$$0,521\pm 0,057$$}\\
       {$$$$}&{$$4$$}&{$$15\pm 1$$}&{$$0,647\pm 0,042$$}&{$$12\pm 1$$}&{$$0,519\pm 0,043$$}\\
       {$$$$}&{$$5$$}&{$$19\pm 1$$}&{$$0,651\pm 0,033$$}&{$$16\pm 1$$}&{$$0,551\pm 0,034$$}\\
       {$$$$}&{$$6$$}&{$$23\pm 1$$}&{$$0,651\pm 0,019$$}&{$$19\pm 1$$}&{$$0,543\pm 0,028$$}\\
       {$$$$}&{$$7$$}&{$$27\pm 1$$}&{$$0,648\pm 0,022$$}&{$$22\pm 1$$}&{$$0,535\pm 0,023$$}\\
       {$$$$}&{$$8$$}&{$$31\pm 1$$}&{$$0,644\pm 0,019$$}&{$$26\pm 1$$}&{$$0,548\pm 0,019$$}\\
       \midrule
       {$$0,00\bar{3}$$}&{$$1$$}&{$$11\pm 1$$}&{$$0,636\pm 0,057$$}&{$$9\pm 1$$} &{$$0,521\pm 0,057$$}\\
       {$$$$}&{$$2$$}&{$$22\pm 1$$}&{$$0,624\pm 0,027$$}&{$$19\pm 1$$}&{$$0,542\pm 0,028$$}\\
       {$$$$}&{$$3$$}&{$$35\pm 1$$}&{$$0,637\pm 0,637$$}&{$$28\pm 1$$}&{$$0,522\pm 0,017$$}\\
       \midrule
       {$$0,001\bar{6}$$}&{$$1$$}&{$$23\pm 1$$}&{$$0,651\pm 0,027$$}&{$$19\pm 1$$}&{$$0,543\pm 0,027$$}\\
       \midrule
       {$$\diameter$$}&{$$$$}&{$$$$}&{$$0.644\pm 0.006$$}&{$$$$}&{$$0.5327\pm 0.0035$$}\\
  
       \bottomrule
      \end{tabular}
    \end{table}



       