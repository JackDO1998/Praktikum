\section{Auswertung}
In diesem Kapitel sollen die aufgenommenen Messwerte Ausgewertet und verrechnet werden.
\subsection{Überprüfung des Reflexionsgesetzes}
Nach dem Reflexionsgesetz ist der Einfallswinkel gleich dem Ausfallswinkel($\alpha_1=\alpha_2$). Es wird
aus verschiedenen definierten Winkeln $\alpha_1$ ein Laserstrahl auf einen Spiegel gerichtet und der 
Reflexionswinkel $\alpha_2$ von einem Schirm abgelesen. Die theoretischen und gemessenen Winkel
sowie die Abweichung sind in \autoref{tab:reflexionsgestz} dargestellt.

\begin{table}
  \centering
    \caption{Einfalls- und Ausfallswinkel eines auf einen Spiegel gerichteten Laserstrahls.}
    \label{tab:ergebnisse}
    \sisetup{table-format=1.2}
    \begin{tabular}{S[table-format=3.2] S S S S S [table-format=3.2]}
      \toprule
      { $\alpha_1$ / °}&{ $\alpha_2$ / °}&{ $\alpha_{2,theo}$ / °}&{ Abweichung / \%}\\
      \midrule
      {$$20$$}  &{$$20\pm 1$$}   &{$$20$$}  &{$$0,0$$} \\
      {$$30$$}  &{$$30\pm 1$$}   &{$$30$$}  &{$$0,0$$}  \\
      {$$35$$}  &{$$35\pm 1$$}   &{$$35$$}  &{$$0,0$$}  \\
      {$$40$$}  &{$$40\pm 1$$}   &{$$40$$}  &{$$0,0$$} \\
      {$$45$$}  &{$$45\pm 1$$}   &{$$45$$}  &{$$0,0$$}  \\
      {$$50$$}  &{$$50\pm 1$$}   &{$$50$$}  &{$$0,0$$}  \\
      {$$60$$}  &{$$60\pm 1$$}   &{$$60$$}  &{$$0,0$$}  \\
      \bottomrule
    \end{tabular}
  \end{table}
Da der Lasepunkt auf dem Schirm eine Ausdehnung von dem Equivalent eines Grades hat und die 
Skala für Einfalls- und Ausfallswinkel ebenfalls nur gradweise eingeteilt ist können 
die Winkel auch nur auf höchstens $\pm \SI[]{1}[]{°}$ genau bestimmt werden.

\subsection{Brechungsgesetz}
Um den Brechungsindex zu bestimmen wird der Brechungswinkel bestimmt. Der Brechungsindex
kann dann leicht über den Zusammenhang:
\begin{center}
  $n_2=\frac{sin \alpha}{sin \beta}$\\
\end{center}
bestimmt werden. Es werden die Werte für $n_2$ einzeln bestimmt, anschließend über \autoref{eq:Mittelwert}
gemittelt und in \autoref{tab:brechungsgesetz} dargestellt. Die Fehler ergeben sich zunächst über \autoref{eq:gaussfehler} und dann für den Mittelwert über
\autoref{eq:mittelwertfehler}
\begin{table}
  \centering
    \caption{Brechung eines Laserstrahls im Medium.}
    \label{tab:brechungsgesetz}
    \sisetup{table-format=1.2}
    \begin{tabular}{S[table-format=3.2] S S S S S [table-format=3.2]}
      \toprule
      { $\alpha_1$ / °}&{ $\beta$ / °}&{ $n_2$ }\\
      \midrule
      {$$20$$}  &{$$14,0\pm 0,5$$}   &{$$1,413\pm 0,084$$}    \\
      {$$30$$}  &{$$20,5\pm 0,5$$}   &{$$1,427\pm 0,055$$}    \\
      {$$35$$}  &{$$22,5\pm 0,5$$}   &{$$1,499\pm 0,049$$}    \\
      {$$40$$}  &{$$25,5\pm 0,5$$}   &{$$1,493\pm 0,041$$}    \\
      {$$45$$}  &{$$28,0\pm 0,5$$}   &{$$1,506\pm 0,036$$}    \\
      {$$50$$}  &{$$31,0\pm 0,5$$}   &{$$1,487\pm 0,031$$}    \\
      {$$60$$}  &{$$35,5\pm 0,5$$}   &{$$1,491\pm 0,024$$}    \\
      \midrule
      {$$\diameter$$}&{$$$$}&{$$1,474\pm 0,014$$}\\
      \bottomrule
    \end{tabular}
  \end{table}
  Mit dem berechneten Wert für $n$ lässt sich über die folgende Beziehung leicht die Lichtgeschwindigkeit
  $v_2$ im Medium berechnen.
 \begin{center}
   $v_2=\frac{v_1}{n_2}$, $v_1=c=29979\times 10^8 \si[]{m/s}$
 \end{center}
 Diese liegt demnach bei $v_2=(2.034\pm 0.019)\times 10^8\si[]{m/s}$


 \subsection{Planparallele Platten}
 \begin{table}
  \centering
    \caption{Brechung eines Laserstrahls im Medium.}
    \label{tab:versatz}
    \sisetup{table-format=1.2}
    \begin{tabular}{S[table-format=3.2] S S S S S [table-format=3.2]}
      \toprule
      { $\alpha_1$ / °}&{ $\beta_{gemessen}$ / °}&{ $s_1$ /cm }&{ $\beta_{berechnet}$ / °}&{ $s_2$ /cm }\\
      \midrule
      {$$20$$}  &{$$14,0\pm 0,5$$}&{$$0,630\pm 0,116$$}&{$$13,417\pm 0,668$$}   &{$$0,689\pm 0,040$$}    \\
      {$$30$$}  &{$$20,5\pm 0,5$$}&{$$1,031\pm 0,119$$}&{$$19,828\pm 0,655$$}   &{$$1,098\pm 0,049$$}    \\
      {$$35$$}  &{$$22,5\pm 0,5$$}&{$$1,370\pm 0,118$$}&{$$22,899\pm 0,646$$}   &{$$1,331\pm 0,054$$}    \\
      {$$40$$}  &{$$25,5\pm 0,5$$}&{$$1,623\pm 0,120$$}&{$$25,854\pm 0,635$$}   &{$$1,589\pm 0,060$$}    \\
      {$$45$$}  &{$$28,0\pm 0,5$$}&{$$1,937\pm 0,120$$}&{$$28,666\pm 0,623$$}   &{$$1,875\pm 0,067$$}    \\
      {$$50$$}  &{$$31,0\pm 0,5$$}&{$$2,222\pm 0,121$$}&{$$31,312\pm 0,609$$}   &{$$2,194\pm 0,074$$}    \\
      {$$60$$}  &{$$35,5\pm 0,5$$}&{$$2,980\pm 0,120$$}&{$$35,981\pm 0,577$$}   &{$$2,943\pm 0,088$$}    \\
      \midrule
      {$$\diameter$$}&{$$$$}&{$$1.68\pm 0.30$$}&{$$$$}&{$$1,67\pm 0,28$$}\\
      \bottomrule
    \end{tabular}
  \end{table}
  
  \subsection{Das Prisma}
  \begin{table}
    \centering
      \caption{Dispersion eines grünen Laserstrahls im Medium.}
      \label{tab:prismagrun}
      \sisetup{table-format=1.2}
      \begin{tabular}{S[table-format=3.2] S S S S S [table-format=3.2]}
        \toprule
        { $\alpha_{1,grün}$ / °}&{ $\alpha_{2,grün}$ / °}&{ $\beta_{1,grün}$ / ° }&{ $\beta_{2,grün}$ / °}&{ $\delta$ / ° }\\
        \midrule
        {$$30$$}  &{$$81\pm 1$$}& {$$17,929\pm 0,512$$}& {$$42,071\pm 0,513$$}   &{$$51.0\pm 1.414$$}    \\
        {$$35$$}  &{$$68\pm 1$$}& {$$21,023\pm 0,549$$}& {$$38,977\pm 0,549$$}   &{$$43,0\pm 1.414$$}    \\
        {$$40$$}  &{$$61\pm 1$$}& {$$23,605\pm 0,573$$}& {$$36,395\pm 0,573$$}   &{$$41,0\pm 1.414$$}    \\
        {$$55$$}  &{$$43\pm 1$$}& {$$32,440\pm 0,628$$}& {$$27,559\pm 0,628$$}   &{$$38,0\pm 1.414$$}    \\
        {$$60$$}  &{$$38\pm 1$$}& {$$35,312\pm 0,640$$}& {$$24,688\pm 0,639$$}   &{$$38,0\pm 1.414$$}    \\
        {$$65$$}  &{$$34\pm 1$$}& {$$37,706\pm 0,648$$}& {$$22,294\pm 0,672$$}   &{$$39,0\pm 1.414$$}    \\
        {$$70$$}  &{$$31\pm 1$$}& {$$39,549\pm 0,653$$}& {$$20,451\pm 0,653$$}   &{$$41,0\pm 1.414$$}    \\
        \midrule
        {$$\diameter$$}&{$$$$}&{$$$$}&{$$$$}&{$$41,6\pm 1,7$$}\\
        \bottomrule
     \end{tabular}
    \end{table}
    
    

    \begin{table}
     \centering
        \caption{Dispersion eines roten Laserstrahls im Prisma.}
        \label{tab:prismarot}
        \sisetup{table-format=1.2}
        \begin{tabular}{S[table-format=3.2] S S S S S [table-format=3.2]}
          \toprule
          { $\alpha_{1,rot}$ / °}&{ $\alpha_{2,rot}$ / °}&{ $\beta_{1,rot}$ / ° }&{ $\beta_{2,rot}$ / °}&{ $\delta$ / ° }\\
          \midrule
          {$$30$$}  &{$$79\pm 1$$}&{$$18,245\pm 0,517$$}&{$$41,755\pm 0,517$$}   &{$$49,0\pm 1,414$$}    \\
          {$$35$$}  &{$$67\pm 1$$}&{$$21,356\pm 0,552$$}&{$$38,643\pm 0,552$$}   &{$$42,0\pm 1,414$$}    \\
          {$$40$$}  &{$$60\pm 1$$}&{$$24,019\pm 0,577$$}&{$$35,981\pm 0,577$$}   &{$$40,0\pm 1,414$$}    \\
         {$$55$$}  &{$$42\pm 1$$}&{$$33,003\pm 0,630$$}&{$$26,997\pm 0,630$$}   &{$$37,0\pm 1,414$$}    \\
          {$$60$$}  &{$$37\pm 1$$}&{$$35,903\pm 0,642$$}&{$$24,097\pm 0,642$$}   &{$$37,0\pm 1,414$$}    \\
         {$$65$$}  &{$$33\pm 1$$}&{$$38,316\pm 0,650$$}&{$$21,684\pm 0,649$$}   &{$$38,0\pm 1,414$$}    \\
          {$$70$$}  &{$$30\pm 1$$}&{$$40,172\pm 0,655$$}&{$$19,828\pm 0,655$$}   &{$$40,0\pm 1,414$$}    \\
          \midrule
          {$$\diameter$$}&{$$$$}&{$$$$}&{$$$$}&{$$40.4\pm 1.6$$}\\
          \bottomrule
      \end{tabular}
    \end{table}

      
      
  


    