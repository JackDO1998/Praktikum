\section{Auswertung}
\label{sec:Auswertung}
In diesem Kapitel werden alle Mittelwerte und deren Fehler berechnet. 
Dazu wurde Python Numpy benutzt. Diese Mittelwerte sind die anzunehmenden, fehlerbehafteten Größen.
\subsection{Schwingungsdauern $T_1$ und $T_2$ der  freischwingenden Pendel}
Die Schwingungsdauern $T_1$ und $T_2$ sind die Schwingungsdauern des linken bzw. des rechten Pendels ohne das
beide mit einer Feder gekoppelt sind.

\subsubsection{Schwingungsdauern für die Pendellänge $L=0,5m$}
Die Schwingungsdauern $T_1$ und $T_2$ ergeben sich aus dem mit (1) berechneten Mittelwerten
zusammen mit deren über (2) berechneten Fehlern der Mittelwerte zu:
\begin{center}
  $T_1=7.31\pm0.05s$\\
  $T_2=7.282\pm0.034s$
\end{center}


\subsubsection{Schwingungsdauern für die Pendellänge $L=0,75m$}
Die Schwingungsdauern $T_1$ und $T_2$ ergeben sich aus dem mit (1) berechneten Mittelwerten
zusammen mit deren über (2) berechneten Fehlern der Mittelwerte zu:
\begin{center}
  $T_1=8.398\pm0.022s$\\
  $T_2=8.408\pm0.016s$
\end{center}


\subsection{Schwingungsdauer $T_+$  für gleichphasige Schwingungen}
$T_+$ ist die Schwingungsdauer für zwei mit einer Feder gekoppelte Pendel die gleichohasig schwingen.
\subsubsection{Schwingungsdauer für $L=0,5m$}
Die Fehlerbehaftete Größe der Schwingungsdauer $T_+$ ist der Mittlewert nach (1) zusammen 
mit dessen Fehler nach (2):
\begin{center}
  $7.25\pm0.04s$
\end{center}
\subsubsection{Schwingungsdauer für $L=0,75m$}
\begin{center}
 $8.415\pm0.033s$ 
\end{center}

\subsection{Schwingungsdauer $T_-$ für gegenphasige Schwingungen}
$T_-$ ist die Schwingungsdauer für zwei mit einer Feder gekoppelten Pendel die gegenphasig schwingen.
\subsubsection{Schwingungsdauer $T_-$ für $L=0,5m$}
\begin{center}
  $7.128\pm0.032s$
\end{center}
\subsubsection{Schwingungsdauer $T_-$für $L=0,75m$}
\begin{center}
  $8.196\pm0.025s$
\end{center}


\subsection{Schwingungsdauer $T$ und Schwebungsdauer $T_S$}
Die Schwingungsdauer $T$ ist die Schwingungsdauer eines gekoppelten Pendels. Die Schwebungsdauer $T_S$ ist 
die Zeit die ein Pendel braucht um vom Stillstand über eine Schwingungsperiode bis zum erneuten Stillstand benötigt.
\subsubsection{Schwingungsdauer $T$}
Die Schwingungsdauer $T$ ist die Verknüpfung aus dem Mittelwert und dessen Fehler.
Dieser Wert lautet für ein 0,5m Pendel:
\begin{center}
  $6.853\pm0.028s$
\end{center}
und für ein 0,75m Pendel:
\begin{center}
  $8.38\pm0.04s$
\end{center}
\subsubsection{Schwebungsdauer $T_S$}
Auch die Schwebungsdauer ist der Mittelwert und sein Fehler.
Für ein 0,5m Pendel hat sie die Größe:
\begin{center}
  $37,33\pm0.24s$
\end{center}
und für ein 0,75m Pendel:
\begin{center}
  $63,7\pm0,4s$
\end{center}


\subsection{Berechnung des Kopplungsgrades $K$}
Der Kopplungsgrad ist eine spezifische Größe der Feder.
Er berechnet sich mit der Formel aus der Versuchsanleitung
über:
\begin{center}
  $K=\frac{T^2_+-T^2_-}{T^2_++T^2_-}$
\end{center}
Mit folgenden Werten für ein 0,5m Pendel:
\begin{center}
  $T_+=7.25\pm0.04s$\\
  $T_-=7.128\pm0.032s$
\end{center}
folgt:
\begin{center}
  $K=0.017\pm0.007$
\end{center}
Mit diesen Werten für das 0,75m Pendel:
\begin{center}
  $T_+=8.415\pm0.033s$
  $T_-=8.196\pm0.025s$
\end{center}
folgt:
\begin{center}
  $K=0.026\pm0.005$
\end{center}

\subsection{Vergleich von berechneter und gemessener Schwebungsdauer $T_S$}
Die Schwebungsdauer berechnet sich laut Versuchsanleitung über:
\begin{center}
  $T_S=\frac{T_+*T_-}{T_+-T_-}$
\end{center}
Mit folgenden Werten für ein 0,5m Pendel:
\begin{center}
  $T_+=7.25\pm0.04s$\\
  $T_-=7.128\pm0.032s$
\end{center}
folgt:
\begin{center}
  $T_{S_{theorie}}=(4.3\pm1.7)*10^2s$
\end{center}
und für das 0,75m Pendel folgt:
\begin{center}
  $T_+=8.415\pm0.033s$\\
  $T_-=8.196\pm0.025s$\\
  $\Rightarrow T_{S_{theorie}}=(2.5\pm0.6)*10^2s$
\end{center}
Die errechneten Schwebungsdauern sind weit größer als die
gemessenen.

\subsection{Verwendete statistische Formeln}

Der Mittelwert:
\begin{center}
$\bar{x}=\frac{1}{n}\sum\nolimits_{i=0} x_i$ (1)
\end{center}

Die Standardabweichung:
\begin{center}
$\sigma=\sqrt{\frac{\sum(x_i-\bar{x})^2}{n-1}}$
  
\end{center}

Der Fehler des Mittelwertes:
\begin{center}
$\sigma_{\bar{x}}=\frac{\sigma}{\sqrt{n}}$ (2)
  
\end{center}

Für alle Rechnungen wurden die Python Funktionen numpy.meam() und numpy.std().



