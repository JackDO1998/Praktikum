\section{Auswertung}
\label{sec:Auswertung}
In diesem Kapitel werden alle Mittelwerte und deren Fehler berechnet. 
Dazu wurde Python Numpy benutzt. Diese Mittelwerte sind die anzunehmenden, fehlerbehafteten Größen.

Der Mittelwert:
\begin{center}
  \begin{equation}
    \label{eq:Mittelwert}
  \bar{x}=\frac{1}{n}\sum\nolimits_{i=0} x_i
  \end{equation} 
\end{center}

Die Standardabweichung:
\begin{center}
  \begin{equation}
    \label{eq:standardabweichung}
  
    \sigma=\sqrt{\frac{\sum(x_i-\bar{x})^2}{n-1}}
  \end{equation}

  
\end{center}

Der Fehler des Mittelwertes:
\begin{center}
  \begin{equation}
    \label{eq:mittelwertfehler}
    \sigma_{\bar{x}}=\frac{\sigma}{\sqrt{n}}
  \end{equation}

  
\end{center}

Die Gaußsche Fehlerfortpflanzung:
\begin{center}
  \begin{equation}
  \label{eq:gaussfehler}  
  
  $$\sigma_x=\sqrt{(\frac{\partial f}{\partial x_1})^2\sigma_{x_1}^2+(\frac{\partial f}{\partial x_2})^2\sigma_{x_2}^2+...+(\frac{\partial f}{\partial x_n})^2\sigma_{x_n}^2}$$
\end{equation}
\end{center}



\subsection{Schwingungsdauern $T_1$ und $T_2$ der  freischwingenden Pendel}
Die Schwingungsdauern $T_1$ und $T_2$ sind die Schwingungsdauern des linken bzw. des rechten Pendels ohne das
beide mit einer Feder gekoppelt sind.

\subsubsection{Schwingungsdauern für die Pendellänge $L=\SI{0,5}{m}$}
Die Schwingungsdauern $T_1$ und $T_2$ ergeben sich aus dem mit \autoref{eq:Mittelwert} berechneten Mittelwerten
zusammen mit deren über \autoref{eq:mittelwertfehler} berechneten Fehlern der Mittelwerte zu:
\begin{center}
  $T_1=\SI{1.463\pm0.001}{s}$\\
  $T_2=\SI{1.456\pm0.007}{s}$
\end{center}


\subsubsection{Schwingungsdauern für die Pendellänge $L=\SI{0,75}{m}$}
Die Schwingungsdauern $T_1$ und $T_2$ ergeben sich aus dem mit \autoref{eq:Mittelwert} berechneten Mittelwerten
zusammen mit deren über \autoref{eq:mittelwertfehler} berechneten Fehlern der Mittelwerte zu:
\begin{center}
  $T_1=\SI{1.680\pm0.004}{s}$\\
  $T_2=\SI{1.68116\pm0.0032}{s}$
\end{center}


\subsection{Schwingungsdauer $T+$  für gleichphasige Schwingungen}
$T+$ ist die Schwingungsdauer für zwei mit einer Feder gekoppelte Pendel die gleichphasig schwingen.
\subsubsection{Schwingungsdauer für $L=\SI{0,5}{m}$}
\label{sec:T+50}
Die fehlerbehaftete Größe der Schwingungsdauer $T+$ ist der Mittlewert nach \autoref{eq:Mittelwert} zusammen 
mit dessen Fehler nach \autoref{eq:mittelwertfehler}:
\begin{center}
  
  $\SI{1.449\pm0.007}{s}$
\end{center}
\subsubsection{Schwingungsdauer für $L=\SI{0,75}{m}$}
\label{sec:T+75}
\begin{center}
 $\SI{1.683\pm0.007}{s}$ 
\end{center}

\subsection{Schwingungsdauer $T-$ für gegenphasige Schwingungen}
$T_-$ ist die Schwingungsdauer für zwei mit einer Feder gekoppelten Pendel die gegenphasig schwingen.
\subsubsection{Schwingungsdauer $T-$ für $L=\SI{0,5}{m}$}
\label{sec:T-50}
\begin{center}
  $\SI{1.426\pm0.006}{s}$
\end{center}
\subsubsection{Schwingungsdauer $T-$für $L=\SI{0,75}{m}$}
\label{T-75}
\begin{center}
  $\SI{1.639\pm0.005}{s}$
\end{center}


\subsection{Schwingungsdauer $T$ und Schwebungsdauer $T_S$}
Die Schwingungsdauer $T$ ist die Schwingungsdauer eines gekoppelten Pendels. Die Schwebungsdauer $T_S$ ist 
die Zeit die ein Pendel braucht um vom Stillstand über eine Schwingungsperiode bis zum erneuten Stillstand 
benötigt.
\subsubsection{Schwingungsdauer $T$}
Die Schwingungsdauer $T$ ist die Verknüpfung aus dem Mittelwert und dessen Fehler.
Dieser Wert lautet für ein $\SI{0,5}{m}$ Pendel:
\begin{center}
  $\SI{1.371\pm0.006}{s}$
\end{center}
und für ein $\SI{0,75}{m}$ Pendel:
\begin{center}
  $\SI{1.667\pm0.009}{s}$
\end{center}
\subsubsection{Schwebungsdauer $T_S$}
Auch die Schwebungsdauer ist der Mittelwert und sein Fehler nach \autoref{eq:Mittelwert} und \autoref{eq:mittelwertfehler}.
Für ein 0,5m Pendel hat sie die Größe:
\begin{center}
  $\SI{37.33\pm0.24}{s}$
\end{center}
und für ein $\SI{0,75}{m}$ Pendel:
\begin{center}
  $\SI{63,7\pm0,4}{s}$
\end{center}


\subsection{Berechnung des Kopplungsgrades $K$}
Der Kopplungsgrad ist eine spezifische Größe der Feder.
Er berechnet sich mit der Formel aus der Versuchsanleitung
über \autoref{eq:kappa}
Mit den in \autoref{sec:T+50} und \autoref{sec:T-50} berechneten Werten folgt:
\begin{center}
  $K=0.017\pm0.007$
\end{center}
Mit den in \autoref{sec:T+75} und \autoref{sec:T-75} berechneten Werten folgt:

\begin{center}
  $K=0.026\pm0.005$
\end{center}

\subsection{Vergleich von berechneter und gemessener Schwebungsdauer $T_S$}
Die Schwebungsdauer berechnet sich laut Versuchsanleitung über \autoref{eq:TS}:

Mit folgenden Werten für ein $\SI{0,5}{m}$ Pendel:
\begin{center}
  $T_+=\SI{7.25\pm0.04}{s}$\\
  $T_-=\SI{7.128\pm0.032}{s}$
\end{center}
folgt:
\begin{center}
  $T_{S_{theorie}}=\SI{(4.3\pm1.7)*10^2}{s}$
\end{center}
und für das $\SI{0,75}{m}$ Pendel folgt:
\begin{center}
  $T_+=\SI{8.415\pm0.033}{s}$\\
  $T_-=\SI{8.196\pm0.025}{s}$\\
  $\Rightarrow T_{S_{theorie}}=\SI{(2.5\pm0.6)*10^2}{s}$
\end{center}







