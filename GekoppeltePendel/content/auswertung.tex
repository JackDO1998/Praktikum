\section{Auswertung}
\label{sec:Auswertung}

\subsection{Schwingungsdauern $T_1$ und $T_2$ der  freischwingenden Pendel}
\subsubsection{Schwingungsdauern für die Pendellänge $L=0,5m$}
Die Schwingungsdauern $T_1$ und $T_2$ ergeben sich aus dem mit (1) berechneten Mittelwerten
zusammen mit deren über (2) berechneten Fehlern der Mittelwerte zu:
\begin{center}
  $$\\
  $$
\end{center}


\subsubsection{Schwingungsdauern für die Pendellänge $L=0,75m$}
Die Schwingungsdauern $T_1$ und $T_2$ ergeben sich aus dem mit (1) berechneten Mittelwerten
zusammen mit deren über (2) berechneten Fehlern der Mittelwerte zu:
\begin{center}
  $$\\
  $$
\end{center}


\subsection{Schwingungsdauer $T_+$  für gleichphasige Schwingungen}
\subsubsection{Schwingungsdauer für $L=0,5m$}
Die Fehlerbehaftete Größe der Schwingungsdauer $T_+$ ist der Mittlewert nach (1) zusammen 
mit dessen Fehler nach (2)
\subsubsection{Schwingungsdauer für $L=0,75m$}
\subsubsection{Schwingungsdauer $T_-$ für gegenphysige Schwingungen}
\subsubsection{Schwingungsdauer $T_-$ für $L=0,5m$}
\subsubsection{Schwingungsdauer $T_-$für $L=0,75m$}
\subsubsection{Schwingungsdauer $T$ und Schwebungsdauer $T_S$}
\subsubsection{Schwingungsdauer $T$}
\subsubsection{Schwebungsdauer $T$}
\subsection{Berechnung des Kopplungsgrades $K$}
\subsection{Vergleich von berechneter und gemessener Schwebungsdauer $T_S$}
\subsection{Verwendete statistische Formeln}
Der Mittelwert:
\begin{center}
  
\end{center}

Der Fehler des Mittelwertes:
\begin{center}
  
\end{center}

Die Standardabweichung:
\begin{center}
  
\end{center}
Siehe \autoref{fig:plot}!
