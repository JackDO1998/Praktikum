\section{Auswertung}
\label{sec:Auswertung}
In diesem Kapitel werden alle Mittelwerte und deren Fehler berechnet. 
Dazu wurde Python Numpy benutzt. Diese Mittelwerte sind die anzunehmenden, fehlerbehafteten Größen.

Der Mittelwert:
\begin{center}
  \begin{equation}
    \label{eq:Mittelwert}
  \bar{x}=\frac{1}{n}\sum\nolimits_{i=0} x_i
  \end{equation} 
\end{center}

Die Standardabweichung:
\begin{center}
  \begin{equation}
    \label{eq:standardabweichung}
  
    \sigma=\sqrt{\frac{\sum(x_i-\bar{x})^2}{n-1}}
  \end{equation}

  
\end{center}

Der Fehler des Mittelwertes:
\begin{center}
  \begin{equation}
    \label{eq:mittelwertfehler}
    \sigma_{\bar{x}}=\frac{\sigma}{\sqrt{n}}
  \end{equation}

  
\end{center}

Die Gaußsche Fehlerfortpflanzung:
\begin{center}
\begin{equation}
  \label{eq:gaussfehler}  
\sigma_x=\sqrt{(\frac{\partial f}{\partial x_1})^2\sigma_{x_1}^2+(\frac{\partial f}{\partial x_2})^2\sigma_{x_2}^2+...+(\frac{\partial f}{\partial x_n})^2\sigma_{x_n}^2}
\end{equation}
\end{center}



\subsection{Schwingungsdauern $T_1$ und $T_2$ der  freischwingenden Pendel}
Die Schwingungsdauern $T_1$ und $T_2$ sind die Schwingungsdauern des linken bzw. des rechten Pendels ohne das
beide mit einer Feder gekoppelt sind.

\subsubsection{Schwingungsdauern für die Pendellänge $L=\SI{0,5}{m}$}
Die Schwingungsdauern $T_1$ und $T_2$ ergeben sich aus dem mit \autoref{eq:Mittelwert} berechneten Mittelwerten
zusammen mit deren über \autoref{eq:mittelwertfehler} berechneten Fehlern der Mittelwerte zu:
\begin{center}
  $T_1=\SI{1.463\pm0.001}{s}$\\
  $T_2=\SI{1.456\pm0.007}{s}$
\end{center}


\subsubsection{Schwingungsdauern für die Pendellänge $L=\SI{0,75}{m}$}
Die Schwingungsdauern $T_1$ und $T_2$ ergeben sich aus dem mit \autoref{eq:Mittelwert} berechneten Mittelwerten
zusammen mit deren über \autoref{eq:mittelwertfehler} berechneten Fehlern der Mittelwerte zu:
\begin{center}
  $T_1=\SI{1.680\pm0.004}{s}$\\
  $T_2=\SI{1.68116\pm0.0032}{s}$
\end{center}


\subsection{Schwingungsdauer $T+$  für gleichphasige Schwingungen}
$T+$ ist die Schwingungsdauer für zwei mit einer Feder gekoppelte Pendel die gleichphasig schwingen.
\subsubsection{Schwingungsdauer für $L=\SI{0,5}{m}$}
\label{sec:T+50}
Die fehlerbehaftete Größe der Schwingungsdauer $T+$ ist der Mittlewert nach \autoref{eq:Mittelwert} zusammen 
mit dessen Fehler nach \autoref{eq:mittelwertfehler}:
\begin{center}
  
  $\SI{1.449\pm0.007}{s}$
\end{center}
\subsubsection{Schwingungsdauer für $L=\SI{0,75}{m}$}
\label{sec:T+75}
\begin{center}
 $\SI{1.683\pm0.007}{s}$ 
\end{center}

\subsection{Schwingungsdauer $T-$ für gegenphasige Schwingungen}
$T_-$ ist die Schwingungsdauer für zwei mit einer Feder gekoppelten Pendel die gegenphasig schwingen.
\subsubsection{Schwingungsdauer $T-$ für $L=\SI{0,5}{m}$}
\label{sec:T-50}
\begin{center}
  $\SI{1.426\pm0.006}{s}$
\end{center}
\subsubsection{Schwingungsdauer $T-$für $L=\SI{0,75}{m}$}
\label{sec:T-75}
\begin{center}
  $\SI{1.639\pm0.005}{s}$
\end{center}


\subsection{Schwingungsdauer $T$ und Schwebungsdauer $T_S$}
Die Schwingungsdauer $T$ ist die Schwingungsdauer eines gekoppelten Pendels. Die Schwebungsdauer $T_S$ ist 
die Zeit die ein Pendel braucht um vom Stillstand über eine Schwingungsperiode bis zum erneuten Stillstand 
benötigt.
\subsubsection{Schwingungsdauer $T$}
Die Schwingungsdauer $T$ ist die Verknüpfung aus dem Mittelwert und dessen Fehler.
Dieser Wert lautet für ein $\SI{0,5}{m}$ Pendel:
\begin{center}
  $\SI{1.371\pm0.006}{s}$
\end{center}
und für ein $\SI{0,75}{m}$ Pendel:
\begin{center}
  $\SI{1.667\pm0.009}{s}$
\end{center}
\subsubsection{Schwebungsdauer $T_S$}
Auch die Schwebungsdauer ist der Mittelwert und sein Fehler nach \autoref{eq:Mittelwert} und \autoref{eq:mittelwertfehler}.
Für ein 0,5m Pendel hat sie die Größe:
\begin{center}
  $\SI{37.33\pm0.24}{s}$
\end{center}
und für ein $\SI{0,75}{m}$ Pendel:
\begin{center}
  $\SI{63,7\pm0,4}{s}$
\end{center}


\subsection{Berechnung des Kopplungsgrades $K$}
\label{sec:kopplungsgrad}
Der Kopplungsgrad ist eine spezifische Größe der Feder.
Er berechnet sich mit der Formel aus der Versuchsanleitung
über \autoref{eq:kappa}. Der zugehörige Fehler lässt sich mit der Gaußschen-Fehlerfortpflanzung aus 
\autoref{eq:gaussfehler} berechnen. Die dazu nötigen Ableitungen lauten:
\begin{center}
  


  $\frac{\partial K}{\partial T+}=\dfrac{4m^2p}{\left(p^2+m^2\right)^2}$\\
  $\frac{\partial K}{\partial T-}=-\dfrac{4p^2m}{\left(m^2+p^2\right)^2}$
\end{center}
Mit den in \autoref{sec:T+50} und \autoref{sec:T-50} berechneten Werten folgt:
\begin{center}
  $K=0.016\pm0.006$
\end{center}
Mit den in \autoref{sec:T+75} und \autoref{sec:T-75} berechneten Werten folgt:

\begin{center}
  $K=0.026\pm0.005$
\end{center}

\subsection{Berechnung von $w_+$, $w_-$ und $w_S$}
In den folgenden Berechnungen wurde für $g=\SI{9,81}{\frac{m}{s^2}}$ und für K die Größen aus
\autoref{sec:kopplungsgrad} angenommen. Die zugehörigen Fehler lassen sich mit der 
Gaußschen-Fehlerfortpflanzung aus \autoref{eq:gaussfehler} berechnen.

\subsubsection{Berechnung von $w_+$}
$w_+$ berechnet sich mit \autoref{eq:omegaplus} für ein $\SI{0,5}{m}$ Pendel zu:
\begin{center}
  $w_+=\SI{4.429}{\frac{1}{s}}$
\end{center}
und für ein $\SI{0,75}{m}$ Pendel:
\begin{center}
  $w_+=\SI{3.616}{\frac{1}{s}}$
\end{center}
\subsubsection{Berechnung von $w_-$}
$w_-$ berechnet sich mit \autoref{eq:omegaminus} für ein $\SI{0,5}{m}$ Pendel zu:
\begin{center}
  $w_-=\SI{4.4368\pm0.0029}{\frac{1}{s}}$
\end{center}

und für ein $\SI{0,75}{m}$ Pendel:
\begin{center}
  $w_+=\SI{3.6261\pm0.0017}{\frac{1}{s}}$
\end{center}

\subsubsection{Berechnung von $w_S$}
$w_S$ berechnet sich mit \autoref{eq:omegas} für ein $\SI{0,5}{m}$ Pendel zu:
\begin{center}
  $w_S=\SI{-0.0073\pm0.0029}{\frac{1}{s}}$
\end{center}
und für ein $\SI{0,75}{m}$ Pendel:
\begin{center}
  $w_S=\SI{-0.0095\pm0.0017}{\frac{1}{s}}$
\end{center}

\subsection{Vergleich von berechneter und gemessener Schwebungsdauer $T_S$}
Die Schwebungsdauer berechnet sich laut Versuchsanleitung über \autoref{eq:TS}
Ihr Fehler ist nach \autoref{eq:gaussfehler} zu errrechnen dazu werden folgende Ableitungen benötigt:
\begin{center}
  $\frac{\partial T_S}{\partial T+}=-\dfrac{m^2}{\left(p-m\right)^2}$\\
  $\frac{\partial T_S}{\partial T-}=\dfrac{p^2}{\left(m-p\right)^2}}$
\end{center}
\subsubsection{Berechnung des theoriewertes von $T+$}
\label{sec:T+th}
Der Theoriewert $T+$ berechnet sich mit \autoref{eq:tplus} für ein $\SI{0,5}{m}$ Pendel zu:
\begin{center}
  $T+=\SI{1.539}{s}$
\end{center}
und für ein ein $\SI{0,75}{m}$ Pendel:
\begin{center}
  $T+=\SI{1.885}{s}$
\end{center}
\subsubsection{Berechnung des theoriewertes von $T-$}
\label{sec:T-th}
Der Theoriewert $T-$ berechnit sich mit \autoref{eq:tminus} für ein $\SI{0,5}{m}$ Pendel zu:
\begin{center}
  $T-=\SI{1.537\pm0.0010}{s}$
\end{center}
und für ein ein $\SI{0,75}{m}$ Pendel:
\begin{center}
  $T-=\SI{1.880\pm0.0009}{s}$
\end{center}

\subsubsection{Berechnung des Theoriewertes von $T_S$}
mit folgenden Werten aus \autoref{sec:T+th} und \autoref{sec:T-th} für ein $\SI{0,5}{m}$ Pendel:

folgt:
\begin{center}
  $T_{S_{theorie}}=(9\pm4)*10^2 s$
\end{center}
und für das $\SI{0,75}{m}$ Pendel folgt:
\begin{center}
  
  $ T_{S_{theorie}}={(7.2\pm1.3)*10^2 s$
\end{center}







