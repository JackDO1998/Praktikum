\section{Diskussion}
\label{sec:diskussion}
Für die Genauigkeit der jeweiligen Messwerte und gegebenenfalls die Abweichung von den idealen Werten 
sind eine Reihe von Faktoren maßgeblich verantwortlich. \newline 
Zum einen sorgen die Näherungen die beispielsweise für den harmonischen Oszillator angenommen wurden für 
Ungenauigkeiten im Hinblick auf die tatsächlich gemessenen Werte. Denn der Zusammenhang 
$w=\sqrt{\frac{g}{l}}$ vernachlässigt den Luftwiderstand und damit den Dämpfungsterm der 
Differentialgleichung. Im realen System tritt eine gedämpfte Schwingung auf, bei der die 
Periodendauern der späteren Umläufe bereits geringer als der ideale Wert sind, weshalb der 
Realwert vom idealen Wert abweicht. Dies fällt besonders bei den sich stark von einander
unterscheidenden Schwebungsdauern $T_S$auf. Außerdem sorgt die Kleinwinkelnäherung für einen weiteren
 Unsicherheitsfaktor, da diese für größere Anfangsauslenkungen des Pendels ungenauer wird. \newline
Zum anderen sind durch die nicht-elektronische Datenaufnahme eine Reihe von menschlichen Faktoren 
zu berücksichtigen. \newline
Aufgrund der Tatsache, dass die Umlaufzeit per Hand gestoppt wird, ergibt sich die Problematik, 
dass der exakte Punkt des maximalen Auslenkwinkels bzw. im Falle der Schwebungsdauer der Ruheposition 
mit bloßem Auge kaum exakt bemessen werden kann. Auch existiert bei dieser Methodik natürlich eine 
geringfügige Verzerrung der Messergebnisse durch die jeweilige Verzögerung der Reaktionszeit des Stoppenden.
 Zuletzt muss für die Messreihen für gleich- bzw. gegensinnige Schwingungen beachtet werden, dass die 
 manuell eingestellten Anfangswinkel nicht exakt übereinstimmen, und damit die Periodendauern von denen der 
 idealen gleich- oder gegensinnigen Schwingungen abweichen. Auch hebt sich die Reihe für die Periodendauer 
 der Schwebung von den restlichen Messungen ab, da die Große Periodendauer nur einen vollständigen Umlauf 
 erlaubt.
 Dennoch fällt auch auf, das sich die Fehler, des über verschiedene Messwerte berechneten
 Kopplungswertes $K$, überschneiden, das spricht für Messwerte die der Realität nahe kommen.  
\section{Literatur}
\label{sec:literatur}
1. TU Dortmund Versuch 106 Gekoppelte Pendel\\
\section{Anhang}
\label{sec:anhang}
Auf den nächsten beiden Seiten folgen Kopien der Originalmesswerte.