\section{Diskussion}
\label{sec:diskussion}
In diesem Versuch wurde zunächst in \autoref{sec:emskupfer} das Emissionspektrum der
benutzten Röntgenquelle bestimmt. Die berechnete Energie der $K_{\alpha}$-Linie liegt bei
$E_{\alpha}=\SI[]{8181.4462}[]{eV}$ und weicht damit vom Theoriewert [3] $E_{\alpha}=\SI[]{8048.11}[]{eV}$
um etwa 1,6\% ab. Der Wert für die $K_{\beta}$-Linie liegt bei $E_{\beta}=\SI[]{9021.2742}[]{eV}$ 
und weicht damit vom Theoriewert [3] $E_{\beta}=\SI[]{8906.9}[]{eV}$ 
um etwa 1,3\% ab. Im nächsten Schritt wurde dann die wellenlängenabängigie Transmission
berechnet, sie lässt sich schreiben als $T=-0.015*\lambda+(1.225 \pm 0.014)$ hier liegen leider
keine Theoriewerte zum Vergleich vor die Ausgleichsgrade liegt jedoch wie in \autoref{fig:spektrum}
zu sehen sehr gut in den Messpunken und wird eine gute Näherung darstellen. Im darauffolgenden Kapitel
\autoref{sec:compton} wurde dann die Comptonwellenlänge bestimmt. Sie liegt bei $\lambda_c=\SI[]{3.7593}[]{pm}$
und weicht damit um 54,9\% vom bei $\lambda_c,e=\SI[]{2.426}[]{pm}$ liegenden Theoriewert [3] ab.
Damit kann gesagt werden das zumindest der erste Teil des Versuches sehr gut gelungen ist und
recht genaue Ergebnisse mit nur kleinen Abweichungen lieferte. Der letzte Teil leieferte dann e
Ergebnisse die nur die Größenordnung des korrekten Wertes aufzeigen. Hier könnten durch die geringen 
Impulszahlen Fehler entstanden sein. Alles in allem kann von einem gelungenen aufschlussreichen 
Versuch gesprochen werden.

\section{Literatur}
1. TU Dortmund, Versuch Nr. 603 Compton-Effekt \\
2. TU Dortmund, Versuch Nr. 602 Röntgenemmision und Absorbtion\\
3. NIST X-Ray Transition Database, Abgerufen 18.04.2021, von https://physics.nist.gov/PhysRefData/XrayTrans/Html/search.html \\