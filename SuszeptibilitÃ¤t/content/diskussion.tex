\section{Diskussion} % (fold)
\label{sec:Diskussion}
In diesem Versuch sollte zunächst die Güte des verwendeten Selektivverstärkers berechnet werden,
dies war leider nicht möglich da die Ausgangspannung des Sinus Generators nicht bekannt ist, daher
wurde in \autoref{sec:verstaerker} nur die recht ungenaue Verstärkerkurve dargestellt. Es konnten
in der nähe des Maximums der Kurve also bei etwa 35,5KHz keine weiteren Messwerte aufgenommen werden
da der Frequenzgenerator in diesem Bereich die Frequenz nicht halten konnte und so kein eindeutiger 
Messwert, weder von verstärkter Spannung noch von Frequenz, abzulesen war. Die Vermutung liegt nahe das
der Verstärker den Frequenzgenerator in der Nähe der hauptsächlich zu verstärkenden Frequenz beeinflusst
also z.B. mit ihm in Resonanz tritt. Als nächstes wurden dann über Quantenzahlen die Suszeptibilitäten 
bestimmt diese sollten im Rahmen der Präzision der Theorie und abgesehen von Rundungsfehlern genau sein. Die
Abweichungen zu den Literaturwerten kann hier leider nicht gezeigt werden da diese nicht vorliegen.
Um die Suszeptibilitäten zu messen wurde im nächsten Teil eine definierte Menge 
der jeweiligen Probe in eine Lange Spule eingeführt welche Teil einer Brückenspannung
ist. Die Brückenspannung wurde gemessen und abgeglichen und aus der zum Abgleich nötigen änderung der Widerstände
die Suszeptibilität berechnet. Die gemessenen Werte weichen wichen wie in \autoref{tab:vergleich} zu sehen stark von den berechneten ab.
Es sind für Dysprosiumtrioxid 9,03\% und für Gadoliniumtrioxid sogar 69,76\%. 
Mögliche Fehlerquellen sind hier das Glasrohr in welchem die Probe gelagert wurde,
eine nicht bekannte also nur abgeschätzte Raumtemperatur, die nicht ganz Monofrequente verstärkung des Selektivverstärkers
und ungenauigkeiten beim ablesen der Spannungswerte.
Im ganzen kann gesagt werden das die Methode der messung mittels Brückenspannung sehr ungenau und fehleranfällig zu sein scheint.
\section{Literatur}
\label{Literatur}
1. TU Dortmund, Versuch 602 Röntgenemmision und Absorbtion\\
2. Demtröder, Wolfgang, 1995, Experimentalphysik 2, 6.Aufl., Berlin\\
% subsubsection  (end)