\section{Durchführung}
\subsection{Messaparatur}
Für die experimentelle Untersuchung der thermischen Elektronenemmission wird eine Diode entsprechend der schematischen Darstellung verwendet. Die Hochvakuumdiode wird 
von einem Konstantspannungsgerät mit einer Heizspannung und einem Weiteren mit einer Saugspannung versorgt, wobei im zweiten Spannungsgerät zugleich der Anodenstrom
abgelesen werden kann. Für die Messung des Anlaufstromgebietes wird ein Konstantstromgerät für niedrige Spannungen sowie ein separates Strommessgerät verwendet.
\subsection{Messung der Kennlinien}
Zur Messung der Kennlinien wird bei konstanter Heizspannung die Saugspannung variirt und der resultierende Anodenstrom gemessen. Dieses Verfahren wurde für 
fünf Heizspannungen zwischen $2 V$ und $2.5$ durchgeführt. Anschließend wird bei konstanter Heizspannung ein schwaches Gegenfeld ($0-1 V$) angelegt, die entsprechende 
Spannung erhöht und erneut der Anodenstromgemessen, um das Anlaufstromgebiet der Anode zu untersuchen.
