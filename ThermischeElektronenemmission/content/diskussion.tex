\section{Diskussion}
\label{sec:diskussion}
In diesem Versuch wurde zunächst in \autoref{sec:kennlinien} eine Kurvenschar von fünf Kennlinien einer 
Hochvakuumdiode erstellt. Bei den beiden größten Heizleistungen ist nicht mehr zu erkennen ab wann die 
Diodenströme in den Sättigungsbereich kommen da die maximal einstellbare Saugspannung bei $U=250\si[]{V}$
lag. Daher war es auch schwirig in \autoref{sec:exponent} den Gültigkeitsbereich des Langmuir-Schottkyschen-Raumladungsgesetzes
festzulegen und der errechnete Exponent $x=1.450\pm 0.014$ weißt daher gegenüber dem Theoriewert [1] von $x=1,5$ einen Fehler 
von etwa $3.4\si[]{\%}$ auf. 
Im näcsten Kapitel \autoref{sec:anlaufstromgebiet}
wurde dann untersucht wie sich die Diodenströme bei kleinen Gegenfeldern verhalten, um eine Aussage über das
Anlaufstromgebiet machen zu können. Hier wurde an die Messpunkte eine Exponentialfunktion angepasst aus welcher
man die Temperatur $T_{gerechnet}=(1747.32\pm 48.63)\si[]{K}$ ableiten kann. Im darau folgenden Unterkapitel
\autoref{sec:kathodentemperatur} wurden dann für die verschiedenen Sättigungsströme nocheinmal die Kathodentemperaturen 
abgeschätzt. Diese liegen in einem Intervall von  $T_{geschätzt}=1930.045\si[]{K}$ bei einem
Heizstrom von $I_H=2.0\si[]{A}$ bis $T_{geschätzt}=2240.384\si[]{K}$ bei einem Heizstrom von $I_H=2.5\si[]{A}$.
Auffällig ist hier das sich $T_{gerechnet}$ und $T_{geschätzt}$ bei einem Heizstrom von $I_H=2.5\si[]{A}$ um etwa
$493\si[]{K}$ also rund $28.2\si[]{\%}$ voneinander unterscheiden. Dies ist wahrscheinlich darauf zurückzuführen
das sich die Heizspannung am Netzgerät nur schlecht ablesen ließ, diese in der Rechnung aber ein wichtiger Faktor ist.
Im letzten Unterkapitel des Auswertungsteils \autoref{sec:arbeit} wurde dann die Austrittsarbeit aus der Wolframelektrode
berechnet diese liegt im Mittelwert bei $W=(6.47\pm 0.09)\si[]{eV}$ und weicht damit um etwa $42.19\si[]{\%}$ von
dem Literaturwert [2] ($W=\SI[]{4.55}[]{eV}$) ab. Dies wird vorallem daran liegen das die zur berechnung wichtigen Kathodentemperaturen
in \autoref{sec:kathodentemperatur} nur abgeschätzt wurden und diese Abschätzung wie oben geschrieben eher ungenau
zu sein scheint. Auch wenn die berechneten Daten eher große Abweichungen zum jeweiligen Soll aufweisen, kann
von einem gelungenen und aufschlussreichen Versuch gesprochen werden.

\newpage
\section{Literatur}
[1] TU-Dortmund, Physikalisches Praktikum, Versuch Nr. 504 Thermische Elektronenemmission\\
[2] Hütte, Das Ingenieurwesen 34. Auflage, Tabelle 16-6, Springer Vieweg 2012

\section{Anhang}
Auf den nächsten Seiten befindet sich ein Scan der Originalmesswerte.