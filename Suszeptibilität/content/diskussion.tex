\section{Diskussion} % (fold)
\label{sec:Diskussion}
In diesem Versuch sollte zunächst die Güte des verwendeten Selektivverstärkers berechnet werden,
sie liegt bei $Q=51.02$ und weicht somit um etwa 96\% vom Theoriewert, $Q=100$, aus Quelle 1. ab. In
\autoref{sec:verstaerker} ist die gemessene Verstärkerkurve mit den linearen interpolationen 
zwischen den Messwerten dargestellt. Es konnten
in der nähe des Maximums der Kurve also bei etwa $\SI[]{35.5}[]{kHz}$ keine weiteren Messwerte aufgenommen werden
da der Frequenzgenerator in diesem Bereich die Frequenz nicht halten konnte und so kein eindeutiger 
Messwert, weder von verstärkter Spannung noch von Frequenz, abzulesen war. Die Vermutung liegt nahe das
der Verstärker den Frequenzgenerator in der Nähe der hauptsächlich zu verstärkenden Frequenz beeinflusst
also z.B. mit ihm in Resonanz tritt. Als nächstes wurden dann über Quantenzahlen die Suszeptibilitäten 
bestimmt diese sollten im Rahmen der Präzision der Theorie und abgesehen von Rundungsfehlern genau sein. Die
Abweichungen zu den Literaturwerten kann hier leider nicht gezeigt werden da diese nicht vorliegen.
Um die Suszeptibilitäten zu messen wurde im nächsten Teil eine definierte Menge 
der jeweiligen Probe in eine Lange Spule eingeführt welche Teil einer Brückenspannung
ist. Die Brückenspannung wurde gemessen und abgeglichen und aus der zum Abgleich nötigen Änderung der Widerstände
die Suszeptibilität berechnet. Die gemessenen Werte weichen wie in \autoref{tab:vergleich} zu sehen stark von den berechneten ab.
Es sind für  Gadoliniumtrioxid 17,3\% und für Dysprosiumtrioxid sogar 37.3\%, selbst wenn man den Fehler voll mit einbezieht
gibt es dennoch Abweichungen von etwa 36,4\% und 16,9\%. Für beide Messungen liegt der errechnete Wert also weit außerhalb des Fehlerintervalls.
Ebenfalls wurde dann die  Suszeptibilität mithilfe der nach dem einführen der Probe gemessenen Brückenspannung bestimmt.
Die Ergebmisse liegen hier bei $\chi_{Dy2 O_3}=(40.6\pm 0.4) \times 10^{-3}$ und 
$\chi_{Gd2 O_3}=(20.2\pm 0.27) \times 10^{-3}$. Der Wert für $\chi_{Dy2 O_3}$ weicht damit vom unteren Ende des Fehlerintervalls um 61,5\% von dem mittels Quantenzahlen
berechneten und um 120\% von der oberen Fehlergrenze des mittels $\Delta R$ berechneten Wertes ab. $Gd_2 O_3$ weicht um 46,6\% von $\chi_{Quantenzahlen}$
und um 71.36\% von $\chi_{Brückenschaltung}$ ab. Es ist also zu erkennen das sich auch hier die Fehlerintervalle nicht überschneiden. Eine Einschätzung über die Richtigkeit
der Werte ist also kaum möglich.
Mögliche Fehlerquellen sind hier das Glasrohr in welchem die Probe gelagert wurde,
eine nicht bekannte also nur abgeschätzte Raumtemperatur, die nicht ganz monofrequente Verstärkung des Selektivverstärkers
und Ungenauigkeiten beim ablesen der Spannungswerte.
Im ganzen kann gesagt werden, dass die Methode der Messung mittels Brückenspannung sehr ungenau und fehleranfällig zu sein scheint.
\section{Literatur}
\label{Literatur}
1. TU Dortmund, Versuch 606 Röntgenemmision und Absorption\\
2. Demtröder, Wolfgang, 1995, Experimentalphysik 2, 6.Aufl., Berlin\\
% subsubsection  (end)