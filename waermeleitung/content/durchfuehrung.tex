\section{Durchführung}
Für den Versuch wird die Apparatur aus Abbildung \ref{fig:Aufbau} verwendet. Sie besteht aus je einem Stab aus Aluminium und Edelstahl, sowie zwei Stäben aus Messing. Diese können mithilfe des Peltier-Elementes in der Mitte erhitzt oder abgekühlt werden. An zwei Stellen eines jeden Stabes befinden sich Thermoelemente, mit denen 
die Temparatur gemessen und auf einem Datenlogger dargestellt werden kann.
\begin{figure} [h]
    \centering
    \includegraphics[width=10cm, keepaspectratio]{Wärmeleitung Aufbau}
    \caption{Apparatur zur Messung der Wärmeleitung}
    \label{fig:Aufbau}
 \end{figure}
\subsection{statische Methode}
Für die statische Methode wird die Spannung des Peltier-Elements auf $U=5$ V eingestellt. Anschließend wird die Temparatur an den Thermoelementen über eine Zeitspanne von $700$ s aufgezeichnet. Dabei werden $5$ Werte pro Sekunde aufgezeichnet.
\subsection{dynamische Methode}
Bei der dynamischen Methode wird die Betriebsspannung auf $U=8$ V eingestellt. Anschließend werden die Stäbe in $40$ s Abständen abwechselnd geheizt und gekühlt, sodass eine Temparaturwelle der Periode $T=80$ s zustande kommt. Für diese wird über die Dauer von $10$ Perioden die Temparatur bei $2$ Werten pro Sekunde aufgezeichnet. Anschließend wird das gleiche Verfahren für die Periodendauer $T=200$ s über eine länge von sechs Perioden wiederholt.
