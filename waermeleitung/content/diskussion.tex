\section{Diskussion}
In diesem Versuch sollen die Wärmeleitfähigkeiten $\kappa$ von den Metallen Aluminium, Edelstahl und Messing
berechnet werden. Dazu werden zunächst in \autoref{sec:auswertung1} die Temperaturen am weiter vom Heiz- bzw.
Kühlelement entfernten Thermoelement  nach einer Zeit von $t=\SI[]{700}[]{s}$ gemessen. Dabei kommt heraus das 
Aluminium mit $T_A=\SI[]{325.28}[]{K}$ scheinbar die beste wärmeleitfähigkeit besitzt. Darauf folgen, in absteigender
Rheienfolge, der breite Messingstab $T_{Mb}=\SI[]{322.68}[]{K}$, der schmale Messingstab $T_{Ms}=\SI[]{320.10}[]{K}$
und der Edelstahlstab $T_E=\SI[]{311.04}[]{K}$. Daran anschließend werden die Wärmeströme für alle Metallstäbe zu fünf
verschiedenen Zeitpunkten berechnet und in \autoref{tab:strom} dargestellt. Im mittel betragen die Wärmeströme für Aluminium
$(\frac{\Delta Q}{\Delta t})=(0.0042\pm 0.0006)\si[]{Wm}$, für Edelstahl $(\frac{\Delta Q}{\Delta t})=(0.00527\pm 0.00021)\si[]{Wm}$, 
für den breiten Messingstab$(\frac{\Delta Q}{\Delta t})=(0.0057\pm 0.0009)\si[]{Wm}$ und $(\frac{\Delta Q}{\Delta t})=(0.0082\pm 0.0008)\si[]{Wm}$
für den schmalen Messingstab. Es liegen leider keine Literaturwerte vor um einen Vergleich anzustellen. Im nächsten 
Kapitel \autoref{sec:auswertung2}, werden dann mit dem sogenannten Angström-Messverfahren durch periodisches heizen 
und kühlen des jeweiligen Stabendes die Wärmeleitfähigkeiten $\kappa$ berechnet. Diese liegen bei $\kappa_{Messing}=4367.46\si[]{\frac{W}{K*m}}$,
$\kappa_{Aluminium}=18745.14\si[]{\frac{W}{K*m}}$,  $\kappa_{Edelstahl}=(-2.3\pm 1.1)\times 10^{4}\si[]{\frac{W}{K*m}}$. 
Die gemessenen Werte weichen alle um Faktoren von etwa 10 bis 100 ab, sodass ein Vergleich mit den Literaturwerten 
[2.] keinen Sinn ergibt. Dieser  große Fehler welcher auch für Edelstahl auserhalb der berechneten ungenauigkeit 
liegt lässt auf einen unentdeckten Rechenfehler schließen. Weitere kleinere Ungenaigkeiten können darin begründet 
liegen, dass die Isolierung der Stäbe unzureichen ist und nur die Stäbe gegen die Außenwelt nicht aber untereinander
abschirmt und es somit durch Strahlung und Konvektion zum Wärmeaustausch zwischen den beiden Stäben kommen kann.

\section{Literatur}
1. TU-Dortmund, V204: Wärmeleitung von Metallen\\
2. Dr.Jörg Wittrock, Wärmeleitfähigkeit der Elemente,\hyperlink{http://www.wittrock-web.de/pse_leit_th.html}{www.wittrock-web.de} abgerufen am 13.06.2021