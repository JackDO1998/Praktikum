\section{Durchführung}
Für diesen Versuch wird ein Ultrachall-Doppler Generator, eine $2$ MHz Ultraschallsonde sowie ein Rechner verwendet, durch den die Datenaufnahme realisiert wird. Untersucht wird eine Flüssigkeit, die sich aus Wasser, Glycerin und Glaskugeln besteht. Sie wird mithilfe einer Pumpe mit regelbarer Drehzahl durch Rohre verschiedener Durchmesser befördert. Um einen konstanten und reproduzierbaren Winkel des Ultraschalls zur Flüssigkeit zu gewährleisten, werden auf den jeweiligen Rohrdurchmesser abgestimmte Doppler-Prismen mit drei Prismen-Winkeln $\Theta$ ($15^\circ$, $30^\circ$, $60\circ$) verwendet. Aus dem Prismenwinkel lässt sich durch
\begin{equation}
\alpha =90^\circ-\arcsin(\sin(\Theta)\frac{c_L}{c_P})
\end{equation}
der Dopplerwinkel mit den Schallgeschwindigkeiten $c_L$ und $c_P$ in Dopplerflüssigkeit bzw. Prismenmaterial (Akryl) bestimmen\\
\subsection{Messung der Strömungsgeschwindigkeit}
Für einen Rohrdurchmesser wird bei fester Strömungsgeschwindigkeit jeweils für alle drei am Prisma einstellbaren Winkel die Frequenzverschiebung sowie die Intensität gemessen. Dieser Vorgang wird für fünf verschiedene Strömungsgeschwindigkeiten wiederholt.
\subsection{Messung des Strömungsprofils}
Für das Rohr mit Durchmesser  $d=3/8$ Zoll wird das Strömungsprofil aufgezeichnet. Dafür wird bei einem konstanten Prismenwinkel von $\Theta=30^\circ$ die Messtiefe innerhalb des Rohres variirt und erneut Frequenzverschiebung und Intensität gemessen.
